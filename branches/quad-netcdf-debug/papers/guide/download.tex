\section{Downloading and installing SUNTANS}

\subsection{Downloading and installing the latest source} \label{sec:download}

To obtain a copy of SUNTANS, you can either download the latest tarball \verb+suntans.tgz+ from
the suntans web page, or download the latest version from the cvs server
using the directions in Section \ref{sec:cvs}.  If you have a working key, you should
be able to download the latest source with
\begin{verbatim}
cvs checkout suntans
\end{verbatim}
which will download the following three directories (not including the \verb+CVS+ directory):
\begin{itemize}
\item \verb+suntans/main+: Contains the main source code and some examples in \verb+main/examples+.
\item \verb+suntans/mfiles+: Some useful m-files for use with matlab.
\item \verb+suntans/papers+: Contains this user guide.
\end{itemize}
These same directories will be created if you download and unpack \verb+suntans.tgz+ with \verb+tar xzvf suntans.tgz+.

If you have the GNU C compiler installed (\verb+gcc+) you should be able to enter \verb+suntans/main+ and
compile SUNTANS with
\begin{verbatim}
make
\end{verbatim}
Otherwise, you will need to specify the correct C compiler with the variable \verb+CC+ in\\
\verb+suntans/main/Makefile+.  With an appropriate C compiler, the example in \\
\verb+suntans/main/examples/iwaves+ should also compile without any further
installation (after changing \verb+CC+ in \verb+suntans/main/examples/iwaves/Makefile+ if necessary).  
To run the example, enter that directory and type
\begin{verbatim}
make test
\end{verbatim}
This will run an internal waves example on a one-dimensional grid of equilateral
triangles.  To view the results, return to \verb+suntans/main+ and compile the graphical
user interface with
\begin{verbatim}
make sunplot
\end{verbatim}
This will create the \verb+sunplot+ executable, which can then be used to view the
results of the internal waves example with
\begin{verbatim}
./sunplot --datadir=examples/iwaves/data
\end{verbatim}
Note:  This GUI requires the
existence of the \verb+Xlib+ libraries and it is assumed that these are located
in \verb+/usr/X11R6+.  Make sure the \verb+XINC+ and \verb+XLIBDIR+ variables
are specified correctly in the \verb+suntans/main/Makefile+ if \verb+sunplot+ does not compile.

In order to run the other examples (as described in Section \ref{sec:examples}), the
grid generation package Triangle~\cite{TRIANGLE[1996]} must be installed, and to
run them in parallel, the message-passing
interface (MPI) and the parallel graph partitioning libraries (ParMetis~\cite{PARMETIS[1998]})
must be installed. Instructions for 
downloading and installing these packages are available from the individual websites
for each package:
\begin{tabbing}
MPI\hspace{0.5in}\=  \verb+http://www-unix.mcs.anl.gov/mpi/mpich/+\\
ParMetis \> \verb+http://www-users.cs.umn.edu/~karypis/metis/parmetis/+\\
Triangle \> \verb+http://www-2.cs.cmu.edu/~quake/triangle.html+
\end{tabbing}
Note that you must compile the triangle libraries as object files by making them
with \verb+make trilibrary+.  

\subsubsection{Notes on Parmetis}

Currently SUNTANS compiles and runs with both Parmetis-2
and Parmetis-3.  If you are using Parmetis-3 you will need to edit suntans/main/partition.c
to include the correct parmetis header file.  For Parmetis-2, the header file is parmetis.h,
while for version 3 it is parmetislib.h.

\subsubsection{Notes on mpich}

SUNTANS works with both mpich-1 and mpich-2 and macro definitions are used to automatically
select the applicable version, although profiles.c-mpich1 and profiles.c-mpich2 are files known
to work for these respective versions for mpich. 

\medskip

After installing the above software, edit \verb+suntans/main/Makefile.in+ 
so that the directories containing the appropriate packages
are correctly specified as follows:
\begin{itemize}
\item \verb+MPIHOME+ should contain the base directory of the mpich distribution.
For example, 
\verb+MPIHOME=/usr/local/mpich-1.2.7+
\item \verb+PARMETISHOME+ should contain the base directory of the ParMetis distribution.
\item \verb+TRIANGLEHOME+ should contain the base directory of the Triangle libraries.
\end{itemize}
Note that there cannot be any spaces between the ``='' sign and the value.  As an example,
the \verb+Makefile.in+ file might look like
\begin{verbatim}
MPIHOME=/usr/local/mpich-1.2.7
PARMETISHOME=/usr/local/packages/ParMetis-2.0
TRIANGLEHOME=/usr/local/packages/triangle
\end{verbatim}
Once these locations are properly specified, compile the SUNTANS executable in \verb+suntans/main+ and
link it with the software that has been installed with
\begin{verbatim}
make
\end{verbatim}
This will create the main executable \verb+sun+.  
To remove the object files, use
\begin{verbatim}
make clean
\end{verbatim}
To clean up the distribution and return it to the original state it was in upon downloading,
use
\begin{verbatim}
make clobber
\end{verbatim}

The original source for SUNTANS contains empty macro definitions in \\
\verb+suntans/main/Makefile.in+, i.e.
\begin{verbatim}
MPIHOME=
PARMETISHOME=
TRIANGLEHOME=
\end{verbatim}
Undefined macros in this file imply that the software is not installed, and suntans will compile
accordingly.  Note that you cannot run SUNTANS in parallel unless both MPICH and ParMETIS are
installed, since ParMETIS performs the parallel grid partitioning.  However, SUNTANS does not
require the Triangle libraries to run in its serial or parallel modes.  Omission of the Triangle 
libraries requires the generation of grid files using an alternate grid generation package, 
as described in Section  \ref{sec:readgrid}.

\subsection{Keeping up to date with CVS} \label{sec:cvs}

You can keep an up to date copy of the SUNTANS distribution on your machine by using the
cvs repository.  To do so, you need to send your ssh public key to the suntans server administrator so that
it can be installed in the authorized keys file on the suntans server.  To obtain your
public key, use
\begin{verbatim}
ssh-keygen -t rsa
\end{verbatim}
This will create a public key in the file \verb+~/.ssh/id_rsa.pub+, which you should send
via email to the administrator.  You may use an empty passphrase but it is not recommended.
Once you send your public key to the suntans cvs administrator, in the \verb+bash+ shell, type
\begin{verbatim}
export CVSROOT=:ext:cvsuser@suntans:/home/cvs
export CVS_RSH=ssh
\end{verbatim}
This sets the \verb+CVSROOT+ environmental variable so that when you use the cvs commands
they look for the cvs repository on the suntans server as user cvsuser.
You will be granted read-only access to the cvs repository, so you can keep an up to date
copy of the latest source on your machine.  To check out the latest copy, type
\begin{verbatim}
cvs checkout suntans
\end{verbatim}
This will create the \verb+suntans+ directory and all of the subdirectories on the server.
If you would like to sync your copy of SUNTANS with the copy on the server, use
\begin{verbatim}
cvs update suntans
\end{verbatim}
Note that you are only allowed read access to the cvs server.  Any changes you make to
the SUNTANS files on your machine will not be updated on the server unless you are added
to the group list on the server.  You will also not be able to ssh into the server if you
try, as this will freeze your screen as only cvs access is allowed via ssh.
