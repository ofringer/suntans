\section{Specifying initial and boundary conditions} 

Initial and boundary conditions are specified by editing the files \verb+initialization.c+ and \verb+boundaries.c+.  When you
edit either one of these files, you must recompile the SUNTANS executable in order for it to reflect changes made
to the initial or boundary conditions.

\subsection{Initial conditions}

Initial conditions are set by altering the functions in the file \verb+initialization.c+ so that they return the
desired initial distributions.  There are five functions that specify the initial depth, free-surface,
salinity, temperature, and velocity fields.  Each function takes as its argument the x,y,z coordinates of
the desired initial condition.  You need to edit the function to return the desired initial value based on the
given x,y,z coordinates.  The five functions are
\begin{itemize}
\item \verb+ReturnDepth+ This returns the depth as a function of the given x,y coordinate. 
As an example, to set a linear slope that rises from 10 m depth to 1 m depth in 1000 m, the
function would look like
\begin{verbatim}
REAL ReturnDepth(REAL x, REAL y) {
  return 10 - 9*x/1000;
}
\end{verbatim}
Note that the depth is specified as a positive quantity.  This function is only used when the \verb+IntDepth+ variable is
set to \verb+0+ in \verb+suntans.dat+.  Otherwise, when \verb+IntDepth+ is set to 1, 
the depth is interpolated from the file specified by the \verb+depth+
file in \verb+suntans.dat+.  This file must contain the depth data in three columns given by x, y, depth, where depth
is specified as a negative quantity and elevation is positive.
\item \verb+ReturnFreeSurface+ This returns the initial free surface distribution as a function of the x,y coordinate.
As an example, to initialize the free surface with a 1 m cosine seiche in a 1000 m domain, the function would look like
\begin{verbatim}
REAL ReturnFreeSurface(REAL x, REAL y, REAL d) {
  return cos(PI*x/1000);
}
\end{verbatim}
Note that \verb+PI+ is a global variable defined in the file \verb+suntans.h+ as \verb+3.141592654+.  The depth
is also supplied to the \verb+ReturnFreeSurface+ function in case the free surface is a function of the depth.
\item \verb+ReturnSalinity+ and \verb+ReturnTemperature+  These are similar functions in that they return the
specified scalar field as a function of the x, y, and z coordinates.  Note that if \verb+beta+ is 0 in \verb+suntans.dat+,
then salinity transport is not computed, likewise if \verb+gamma+ is 0, then temperature transport is not computed.  
As it is now, the code computes the temperature as a passive scalar while the density anomaly $\rho$ is computed solely as a function of the salinity anomaly $s$, such that
\[\frac{\rho}{\rho_0}=\beta s\,.\]  This equation of state is specified in the file \verb+state.c+.
\item \verb+ReturnHorizontalVelocity+ This returns the horizontal velocity defined at the faces of each cell.  The initial vertical velocity is
computed by continuity from this initial velocity field.  Since this function returns the velocity normal to a cell face, then
you must specify both velocity components \verb+u+ and \verb+v+ and then return the component normal to the face, which is
defined by the vector with components \verb+n1+ and \verb+n2+.  As an example, to return an irrotational vortex with
maximum velocity of 1, centered
about (x,y)=(5,5), the function would appear as
\begin{verbatim}
REAL ReturnHorizontalVelocity(REAL x, REAL y, REAL n1, REAL n2, REAL z) {
  REAL u, v, umag=1;

  u = -umag*(y-5)/5;
  v = umag*(x-5)/5;

  return u*n1+v*n2;
}
\end{verbatim}
\end{itemize}

\subsection{Specifying boundary conditions} \label{sec:boundary}

SUNTANS allows the specification of the velocity at the open boundaries in the file
\verb+boundaries.c+.  Boundary condition types on the velocity field can be specified in the \verb+pslg+ file
which sets the marker as described in Sections \ref{sec:tri} and \ref{sec:readgrid}.  The marker type
can be one of the following:
\begin{itemize}
\item \verb+1+ For closed boundaries.
\item \verb+2+ For open or velocity-specified boundaries.
\item \verb+3+ For open or stage-specified boundaries.
\item \verb+4+ For no-slip boundary conditions.
\end{itemize}
If the marker is set to \verb+1+, then the velocity is set to \verb+0+ at those edges.  If it is set to \verb+2+,
then the velocity is specified in the file \verb+boundaries.c+ by iterating over edges with marker=\verb+2+.
If it is set to \verb+3+ then stage can be specified at the Voronoi cell center for the cell adjacent to this edge. 
No-slip boundaries are specified by setting the marker to \verb+4+, where the velocity on the edge is specified by 
updating \verb+phys->boundary_u+, \verb+phys->boundary_v+, and \verb+phys->boundary_w+.  No-slip conditions on top and bottom cells are 
specified by setting \verb+CdB=-1+ or \verb+CdT=-1+, respectively.  An example covering usage of type \verb+2+ boundary conditions, type 
\verb+3+ boundary conditions, and type \verb+4+ boundary conditions will be covered in turn.

\subsubsection{Open or velocity-specified boundary conditions}
This file loops through the edges whose edge markers are \verb+2+.  As an example, consider the flow in a 
channel that is 1000 m long in which two boundaries are specified as inflow and outflow and the other two 
are specified as solid walls, as shown in Figure \ref{fig:boundaries}.  
\insertfig{.5}{figures/boundaries}{Depiction of a channel with an inflow at $x=0$ m and outflow at $x=1000$ m with
the inflow/outflow boundaries specified with 2 markers and the solid walls with 1 markers.  The indices of
the cells adjacent to the boundaries are denoted by ib while the boundary faces have indices j.}{fig:boundaries}
Suppose you would like to
specify an incoming velocity that oscillates at a frequency $\omega$ due to an incoming wave at $x=0$ m and that
you would like to use the linearized open boundary condition to specify the outflow at $x=1000$ m.  Since the
edge markers at the inflow and outflow are the same, you need to set the boundary condition based on the x,y
location of the specified edge.  To do so, you would set up the functions in the \verb+boundaries.c+ file as follows
\begin{itemize}
\item Function OpenBoundaryFluxes:
\begin{verbatim}

  for(jptr=grid->edgedist[2];jptr<grid->edgedist[3];jptr++) {
    j = grid->edgep[jptr];

    ib = grid->grad[2*j];

    if(grid->xv[ib]>500) {
      for(k=grid->etop[j];k<grid->Nke[j];k++) 
         ub[j][k] = -phys->h[ib]*sqrt(prop->grav/(grid->dv[ib]));
    } else {
      for(k=grid->etop[j];k<grid->Nke[j];k++) 
         ub[j][k]=phys->boundary_u[jptr-grid->edgedist[2]][k]*grid->n1[j]+
                  phys->boundary_v[jptr-grid->edgedist[2]][k]*grid->n2[j];
    }
 }

\end{verbatim}
\item BoundaryVelocities Function:
\begin{verbatim}

  for(jptr=grid->edgedist[2];jptr<grid->edgedist[3];jptr++) 
    j = grid->edgep[jptr];

    ib = grid->grad[2*j];

    if(grid->xv[ib]>500) {
      for(k=grid->etop[j];k<grid->Nke[j];k++) {
        phys->boundary_u[jptr-grid->edgedist[2]][k]=phys->uc[ib][k];
        phys->boundary_v[jptr-grid->edgedist[2]][k]=phys->vc[ib][k];
        phys->boundary_w[jptr-grid->edgedist[2]][k]=
            0.5*(phys->w[ib][k]+phys->w[ib][k+1]);
      } 
    } else {
      for(k=grid->etop[j];k<grid->Nke[j];k++) {
        phys->boundary_u[jptr-grid->edgedist[2]][k]=
            prop->amp*cos(prop->omega*prop->rtime);
        phys->boundary_v[jptr-grid->edgedist[2]][k]=0;
        phys->boundary_w[jptr-grid->edgedist[2]][k]=0;
        } 
    }
 }

\end{verbatim}
\end{itemize}
The outermost \verb+jptr+ loop loops over the edges which have boundary markers specified as \verb+2+.
The \verb+j+ index is an index to an edge along the open boundary, and 
the \verb+ib+ index is an index to the cell adjacent to that boundary edge.  Since the Voronoi points
are specified at the \verb+ib+ indices, then we need to specify the boundary condition based on
the location of the Voronoi points of the adjacent cell.  In this case the open boundary exists at x=1000, but since
we know that this boundary exists for boundary edges for $x>500$, we use the if statement
\verb+if(grid->xv[ib]>500)+ to specify these boundary edges.  The first part of the if statement
sets the flux at the open boundary over all the vertical levels with
\begin{verbatim}
      for(k=grid->etop[j];k<grid->Nke[j];k++) 
        ub[j][k] = -phys->h[ib]*sqrt(prop->grav/(grid->dv[ib]));
\end{verbatim}
which is identical to 
\[ u_b = -h_b\sqrt{\frac{g}{d}} \,,\]
and this is the linearized shallow-water free-surface boundary condition (Note that the upwind depth
and free-surface height are being used to approximate the values at the open boundary).  This loop uses
the \verb+Nke[j]+ variable, which is the number of vertical levels at face \verb+j+,
and \verb+etop[j]+, which is the index of the top level at the boundary.  The variable
\verb+etop[j]+ is usually \verb+0+ unless filling and emptying occur at the boundary
edges.  

In addition to setting
the boundary flux in the OpenBoundaryFluxes function, the user must also set the Cartesian components
of velocity in the BoundaryVelocities function.  These are used to compute the advection of momentum. 
They can also be used to specify boundary velocities for no-slip type \verb+4+ boundary conditions.
At the open outflow boundary, the boundary velocity components are set to the upwind value of the velocity
component.  For the u-component of velocity, for example, the boundary value is specified
with
\begin{verbatim}
phys->boundary_u[jptr-grid->edgedist[2]][k]=phys->uc[ib][k];
\end{verbatim}
Here, \verb+uc+ is the u-component of velocity at the Voronoi point with index \verb+ib+.

At the inlet, only the Cartesian velocity components need to be specified.  In this example,
the u-component of velocity at the inlet is set with the code
\begin{verbatim}
phys->boundary_u[jptr-grid->edgedist[2]][k]=
               prop->amp*cos(prop->omega*prop->rtime);
\end{verbatim}
and the other components are set to 0.  This function uses the variables \verb+amp+ and \verb+omega+
which are set in \verb+suntans.dat+, and \verb+rtime+ is the physical time in the simulation.
The Cartesian velocities which are specified at the inlet boundary are then used to compute
the flux in the OpenBoundaryFluxes function with
\begin{verbatim}
ub[j][k]=phys->boundary_u[jptr-grid->edgedist[2]][k]*grid->n1[j]+
         phys->boundary_v[jptr-grid->edgedist[2]][k]*grid->n2[j];
\end{verbatim}
This is just the dot product of the velocity field specified at the boundary with the
normal vector at the boundary edge, which points into the domain by definition.
Examples of how to employ velocity boundary conditions are described in Sections \ref{sec:boundary_ex}
and \ref{sec:internalwaves}.  Section \ref{sec:boundary_ex} also demonstrates 
how to specify salinity and temperature at the boundaries.

\subsubsection{Open or stage-specified boundary conditions}

A good example of specifying stage boundary conditions (type \verb+3+) is the 
\verb+suntans/main/examples/+ \verb+estuary+ example,
where a stage boundary condition is specified at the ocean boundary to the west of the domain.
A simplified tidal signal is specified via the following code contained within the function 
BoundaryVelocities() of \verb+suntans/main/examples/estuary/boundaries.c+.

\begin{verbatim}
  for(iptr=grid->celldist[1];iptr<grid->celldist[2];iptr++) {
    i = grid->cellp[iptr];
    phys->h[i]=-prop->amp*sin(prop->omega*prop->rtime);
  }
\end{verbatim}

In this example, type \verb+2+ boundary conditions result in indicies to computational cells adjacent
to type \verb+2+ boundary condition being stored in [ \verb+celldist[1]+, \verb+celldist[2]+).  
Stage information for these cells can then be specified by setting the cell stage via the cell pointer
\verb+i=grid->cellp[iptr]+.  In this example, \verb+prop->amp+ specifies tidal amplitude $A$, 
\verb+prop->omega+ specifies the frequency $\omega$, and \verb+prop->rtime+ the simulation time $t$ within
the prototype function $h = -A\sin\left(\omega t\right)$.

\subsubsection{No-slip boundary conditions}

No-slip boundary conditions in the plane XZ are specified via type \verb+4+ boundary conditions
in \verb+edges.dat+. Setting \verb+CdT=-1+ and \verb+CdB=-1+ within \verb+suntans.dat+ 
specifies no-slip boundary conditions for the top and bottom of the domain within the plane XY. 
As an example, consider the \verb+testXZ+ case in \verb+suntans/main/examples/cavity+.  The parameter file
\verb+suntans/main/examples/cavity/rundata/suntansXZ.dat+ has 
\begin{verbatim}
CdT                -1   # Drag coefficient at surface
CdB                -1   # Drag coefficient at bottom
\end{verbatim}
and the function BoundaryVelocities() in \verb+suntans/main/examples/cavity/boundariesXZ.c+ has
\begin{verbatim}
  for(jptr=grid->edgedist[4];jptr<grid->edgedist[5];jptr++) {
    
    j = grid->edgep[jptr];
    ib=grid->grad[2*j];
    boundary_index = jptr-grid->edgedist[2];
    
    for(k=grid->ctop[ib];k<grid->Nk[ib];k++) {
      if(grid->xe[j]<0.5) 
        phys->boundary_w[boundary_index][k]= -1.0;
      else
        phys->boundary_w[boundary_index][k]=0.0;
      phys->boundary_v[boundary_index][k] = 0.0;
      phys->boundary_u[boundary_index][k] = 0;
    }
  }
\end{verbatim}
where the no-slip boundary condition is applied on the west side of the boundary directed 
downward via \verb+phys->boundary_w[boundary_index]= -1.0+ for edges with markers \verb+jptr+ in
[\verb+grid->edgedist[4]+\verb+grid->edgedist[5]+).  

\subsection{Specifying heat flux} \label{sec:heatflux}

To specify source terms in the temperature equation, you will need to edit the
HeatSource() function in suntans/main/sources.c.  The implementation assumes
adiabatic surface and bottom boundary conditions, and so all heat fluxes are implemented
as source terms in the temperature equation of the form
\begin{eqnarray*}
\D{T}{t} + \ub\cdot\nabla T &=& \D{}{z}\left[\left(\kappa_T + \kappa_{tv}\right) \D{T}{z}\right] + Q(T)\,,
\end{eqnarray*}
where $\kappa_T$ is the molecular diffusivity (defined in suntans.dat) and $\kappa_{tv}$ is the
vertical turbulent eddy-diffusivity, computed in turbulence.c.  
terms to simplify the discussion, the temperature equation is given by
\begin{equation}\label{eq:T}
\D{T}{t}  = Q(T)\,.
\end{equation}
The source term can be linearized about some temperature $T_0$ so that, defining $T=T_0 + \Delta T$,
we have
\begin{eqnarray*}
Q(T) &=& Q(T_0 + \Delta T)\,,\\
     &=& Q(T_0) + \left(\D{Q}{T}\right)_{T_0}\Delta T  + O(\Delta T^2)\,,\\
     &=& Q(T_0) + \left(\D{Q}{T}\right)_{T_0}\left(T - T_0\right)  + O(\Delta T^2)\,,\\
     &=& Q(T_0) - \left(\D{Q}{T}\right)_{T_0}T_0 + \left(\D{Q}{T}\right)_{T_0} T + O(\Delta T^2)\,,\\
     &=& A(T_0) + B(T_0) T + O(\Delta T^2)\,,
\end{eqnarray*}
where
\begin{eqnarray*}
A(T_0) = Q(T_0) - \left(\D{Q}{T}\right)_{T_0}T_0\,,\\
B(T_0) = \left(\D{Q}{T}\right)_{T_0}\,.
\end{eqnarray*}
The heat equation is discretized with the theta method so that equation~(\ref{eq:T}) becomes
\begin{eqnarray*}
\frac{T^{n+1} - T^n}{\Delta t} &=& A(T_0) + B(T_0)\left[\theta T^{n+1}+(1-\theta)T^n\right]\,,
\end{eqnarray*}
where $\theta$ is specified by the value of \verb+theta+ in suntans.dat.
After rearranging, the temperature at the new time step is obtained with
\begin{eqnarray*}
T^{n+1} = \left[1 - \theta \Delta t B(T_0)\right]^{-1}\left[T^n + \Delta t A(T_0) 
                  + (1-\theta) \Delta t B(T_0)T^n\right]\,.
\end{eqnarray*}
Typically the best linearization is to linearize about the temperature at time step $n$ so
that $T_0=T^n$. A simple implementation of heat flux is provided in the windstress example
in \verb+suntans/main/examples/windstress+ and the code is in the HeatSource() function in
\verb+suntans/main/examples/windstress/sources.c+.

