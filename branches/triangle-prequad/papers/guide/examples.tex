\section{SUNTANS Examples} \label{sec:examples}

\subsection{Running the examples}

Each example is located in one of the directories in the \verb+examples+ directory in
the main source directory.  As an example, consider the internal waves example located
in \verb+examples/iwaves+.  The internal waves simulation in this directory is defined by the following files
in the \verb+examples/iwaves+ directory:
\begin{itemize}
\item \verb+initialization.c+: Initial conditions
\item \verb+boundaries.c+: Boundary conditions
\item \verb+rundata/suntans.dat+: Parameters
\item \verb+rundata/pslg.dat+: Planar straight line graph
\end{itemize}
All examples are run in the same way.
The makefile in the \verb+iwaves+ directory compiles and links the initial and boundary condition
files with the main sun executable in the source directory.  Then it copies the parameter
file (\verb+suntans.dat+) and the planar straight line graph file to the directory \verb+data+ in
the particular example directory.  Then the main sun executable is run with
\begin{verbatim}
../../sun -t -g -s -vv --datadir=data
\end{verbatim}
where the particular flags may depend on the example, but the executable is run from
the main source directory and the data is local to the particular examples directory.
In order to compile and run an example, enter that example
directory and type \verb+make test+.  This will run the example simulation and place
the output data into the data directory in the examples directory.
While the simulation is running, the data can be viewed from another command line
from the main source directory with
\begin{verbatim}
./sunplot --datadir=examples/iwaves/data
\end{verbatim}
It is important that you run \verb+make clobber+ between successive tests in other
test directories to ensure that you are recompiling the \verb+../../sun+ executable
with the correct initial and boundary condition files each time.  Without clobbering,
rerunning \verb+make test+ will run the test case while only reflecting changes in \verb+suntans.dat+.
Therefore, it is not necessary to clobber if the same test is being run with different 
parameters in \verb+suntans.dat+.  The following test cases exist in the examples
directory.  The first 5 are discussed in detail in what follows.  Note that many examples
directories also contain an \verb+mfile+ directory that has useful mfiles for analyzing output.
\begin{enumerate}
\item tides: Example tidal forcing in Monterey Bay
\item accuracy: Demonstrates second-order time accuracy using an internal waves seiche
\item boundaries: River plume example
\item cavity: Cavity flow example with no-slip boundary conditions
\item lockexchange: Lock exchange flow with closed boundaries
\item iwaves: Internal tide generation over a simplified continental shelf
\item channel: Simple channel flow demonstrating use of the turbulence model
\item cylinder: Formation of vortex street in lee of a cylinder
\item estuary: Tidal flow in a simple estuary with specified river inflow
\item tides-restart: Same as tides, but shows how to use restart filed
\item wetdry: Example of wetting and drying
\item windstress: Constant wind and heat forcing over a parabolic lake
\end{enumerate}

\subsection{Tidal forcing} \label{sec:tidalforcing}

This example is in the \verb+examples/tides+ directory and contains an example of how
to force SUNTANS with tidal constituents at the boundaries.  The domain consists of
Monterey Bay and is a simplified example of the case run by Jachec\etal~\cite{JACHEC[2006]},
which focuses on internal wave generation in the region.  This example consists only of the
barotropic flow and the user should be warned that because of the coarse resolution of the
grid and of the boundary conditions that the results have not been validated.  The important feature of
this example is that it can be used as a starting point from which to set up a more
complex simulation containing tidal forcing.  Note that all files and directories are relative
to the directory \verb+suntans/main/examples/tides+ unless otherwise noted.  A very brief
README file also exists in \verb+suntans/main/examples/tides/README+ for the impatient.

\subsubsection{Grid and bathymetry}

The grid for this example was generated with GAMBIT and is shown in Figure \ref{fig:mbaygrid}
(for tips on using GAMBIT for SUNTANS please see\\
\verb+http://suntans.stanford.edu/documentation/suntans_tutorial.pdf+).
Grids like this can be generated by obtaining a coastline from the NOAA coastline extractor \\
\verb+http://rimmer.ngdc.noaa.gov/coast/+\\
and reading it into a grid generation program {\bf after} converting the coordinates to a Cartesian grid.
Most grids in SUNTANS are obtained with a UTM projection, but for larger grids that cross over multiple
UTM zones, the Mercator projection is also suitable.  
Upon running SUNTANS (with \verb+-g -vv+), Voronoi distance statistics will be output as follows:
\begin{verbatim}
Voronoi statistics:
        Minimum distance: 1.97e+02
        Maximum distance: 4.43e+03
        Mean distance: 2.85e+03
        Standard deviation: 4.14e+02
\end{verbatim}
Note that while the minimum distance in this example is acceptable, the Voronoi distances do
not indicate whether there are degenerate cells in the domain (i.e. obtuse triangles).  In order
to correct for any degenerate triangles, the parameter \verb+CorrectVoronoi+ is set to \verb+-1+,
and the parameter \verb+VoronoiRatio+ is set to \verb+85+ degrees (these parameters are in \verb+suntans.dat+).  
This corrects any triangles
with angles greater than 85 degrees by placing the Voronoi points at the triangle centroids.
After correction, the statistics are displayed as
\begin{verbatim}
Corrected 11 of 1534 cells with angles > 85.0 degrees (0.72%).
Voronoi statistics after correction:
        Minimum distance: 9.49e+02
        Maximum distance: 4.34e+03
        Mean distance: 2.85e+03
        Standard deviation: 3.95e+02
\end{verbatim}
This highlights the fact that 11 cells in the grid were obtuse and were corrected.  Also note that
as a result of the correction the minimum distance between Voronoi points increased from 197 m to
949 m.  This substantially raises the minimum allowable time step for stability of the nonlinear
advection terms.
\insertfig{0.75}{figures/mbaygrid}{Monterey Bay grid generated with GAMBIT.  The boundary edges
with markers of type 2 are highlighted in bold.  The other boundary edges are type 1 (closed).}{fig:mbaygrid}

For the bathymetry, 250 m resolution bathymetry is available in the file \\
\verb+rundata/mbay_bathy.dat+ which was obtained from the MBARI multibeam survey \\
\verb+http://www.mbari.org/data/mapping/monterey/default.htm+\\
This file is specified by the \verb+depth+ variable in \verb+suntans.dat+.
The interpolated bathymetry is depicted in Figure \ref{fig:mbaydepth}.
Upon running SUNTANS, bathymetry from this file is interpolated onto the (possibly corrected) Voronoi points.
Because this procedure can take quite some time, it is a good idea to only run it when necessary and use
the pre-interpolated bathymetry for subsequent runs.  Each time SUNTANS interpolates bathymetry, it outputs
the interpolated bathymetry into the file \verb+mbay_bathy.dat-voro+ (it appends ``-voro'' to the specified
bathymetry file).  This file is created when the variable \verb+IntDepth+ is set to 1 in \verb+suntans.dat+.
Otherwise, if this variable is set to 2, data is not interpolated but instead is read in from the \verb+-voro+ file.
Note that any time changes are made to the Voronoi points, the bathymetry should be re-interpolated. 
\insertfig{0.75}{figures/mbaydepth}{MBARI bathymetry (in m) interpolated onto the Monterey Bay grid.}{fig:mbaydepth}

\subsubsection{Initial conditions}

Initial conditions for this problem are a quiescent free surface and velocity field, and no stratification.  
Stratification can be turned off by setting \verb+beta+ to zero in \verb+suntans.dat+.  This is because the equation
of state, as given in \verb+suntans/main/state.c+ is linear and only requires \verb+beta+ to convert from
salinity to density.  Note that when \verb+beta+ is zero, advection of salinity is also not computed, which can
save on computation time.

\subsubsection{Boundary conditions} \label{sec:tides_bc}

Boundary conditions are given by the first eight tidal constituents as computed by OTIS which can
be obtained from\\
\verb+http://www.coas.oregonstate.edu/research/po/research/tide/+\\
Tidal constituent data is read into SUNTANS in the function \verb+BoundaryVelocities+ in the file 
\verb+boundaries.c+, and the data is stored in the files specified by the variable
\verb+TideInput+ in \verb+suntans.dat+ (the default is \verb+tidecomponents.dat+).  Each processor
will require a file with the name \verb+tidecomponents.dat.proc+, where \verb+proc+ is the processor
number.  This file contains the tidal constituents for the horizontal velocity components and
the free surface height (in cm s$^{-1}$ and cm) at the locations of the centers of each boundary edge of type 2.
These locations are given in the files specified by the variable \verb+TideOutput+ in \verb+suntans.dat+
(the default is \verb+tidexy.dat+). The \verb+TideInput+ file has the following binary format for each processor:
{\footnotesize
\begin{verbatim}
numtides (1 X int) number of tidal constituents.
numboundaryedges (1 X int) number of boundary edges.
omegas (numtides X REAL) frequencies of tidal components.
u_amp[0] (numtides X REAL) amplitude of easting velocity at location 0.
u_phase[0] (numtides X REAL) phase of easting velocity at location 0.
v_amp[0] (numtides X REAL) amplitude of northing velocity at location 0.
v_phase[0] (numtides X REAL) phase of northing velocity at location 0.
h_amp[0] (numtides X REAL) amplitude of free surface at location 0.
h_phase[0] (numtides X REAL) phase of free surface at location 0.
u_amp[1] (numtides X REAL) amplitude of easting velocity at location 1.
u_phase[1] (numtides X REAL) phase of easting velocity at location 1.
v_amp[1] (numtides X REAL) amplitude of northing velocity at location 1.
v_phase[1] (numtides X REAL) phase of northing velocity at location 1.
h_amp[1] (numtides X REAL) amplitude of free surface at location 1.
h_phase[1] (numtides X REAL) phase of free surface at location 1.
...
u_amp[numboundaryedges-1] (numtides X REAL) amplitude of easting velocity at last location.
u_phase[numboundaryedges-1] (numtides X REAL) phase of easting velocity at last location.
v_amp[numboundaryedges-1] (numtides X REAL) amplitude of northing velocity at last location.
v_phase[numboundaryedges-1] (numtides X REAL) phase of northing velocity at last location.
h_amp[numboundaryedges-1] (numtides X REAL) amplitude of free surface at last location.
h_phase[numboundaryedges-1] (numtides X REAL) phase of free surface at last location 
\end{verbatim}
}
The user can create this file or it can be created using the \verb+suntides+ matlab package,
(located in \verb+suntans/mfiles/suntides+)
which is an adaptation of the \verb+tidegui+ package created by Brian Dushaw\footnote{
http://909ers.apl.washington.edu/~dushaw/tidegui/tidegui.html}.
The \verb+suntides.m+ file reads in the x-y locations of the boundary points for each processor
and outputs the associated tidal components in binary form.  The matlab script \verb+tides.m+ 
in the tides example directory performs this action.  The x-y location data files are created
by running SUNTANS, which will complain if no \verb+tidecomponents.dat.proc+ files are found
in the data directory, i.e the directory specified in the command line call in suntans with
\verb+--datadir=+.  In the present example, this directory is given by 
\verb+data+.  Upon running the test, SUNTANS will complain with
\begin{verbatim}
Error opening data/tidecomponents.dat.0!
Writing x-y boundary locations to data/tidexy.dat.0 instead.
Error opening data/tidecomponents.dat.1!
Writing x-y boundary locations to data/tidexy.dat.1 instead.
\end{verbatim}
This indicates that the tidal component data files were missing, so SUNTANS writes the locations
of the markers of type 2 (the boundary edges) to the \verb+tidexy.dat.proc+ files.  Once these
files have been created, the \verb+tides.m+ script can be run to create the \verb+tidecomponents.dat.proc+ files.
This can be run from the command line in unix with \verb+matlab < tides.m+.

In order to prevent transient oscillations associated with the impulsive starting of the tidal
boundary conditions, it is always a good idea to spin-up the boundary conditions over a day or so
with 
\[
F_{actual} = F_{imposed}\left[1-\exp(t/\tau_{ramp})\right]\,,
\]
so that the forcing approaches the imposed forcing over a time scale of $\tau_{ramp}$.  This is
implemented in the file \verb+boundaries.c+ in the function \verb+BoundaryVelocities+,
where the tidal data is input.  For example, the u-velocity component is imposed with the line
\begin{verbatim}
phys->boundary_u[jind][k]=u*(1-exp(-prop->rtime/prop->thetaramptime))/100.0;
\end{verbatim}
Note that the value of \verb+u+ is multiplied by the ramping function as well as divided by
100.0 in order to convert from cm s$^{-1}$ to m s$^{-1}$.  The ramping function ramps up over
a time scale given by the variable \verb+thetaramptime+, which is defined as \verb+86400+ (one day in seconds)
in \verb+suntans.dat+.  This variable also implies more damping during this initial transient period by
ramping the value of \verb+theta+ in the simulation from 1 (fully implicit barotropic time-stepping) to
the value specified in \verb+suntans.dat+ by the variable \verb+theta+.  Note that $\theta>0.5$ for stability,
and the value of $\theta=0.55$ is typically used.  In the above expression for the damping transient,
for the time, the variable \verb+rtime+ is used, which represents the time, in seconds, from the beginning
of the simulation, which starts at \verb+rtime=0+.  Using the \verb+suntides+ example, all tidal phases are with
reference to the beginning of the year specified when running the \verb+suntides.m+ script.  For the
present example, the year is given by the line \verb+year = 2006+ in \verb+setup_tides.m+.
In order to start the simulation from a specific day in 2006, a time offset is specified in \verb+suntans.dat+
as the variable \verb+toffSet+, which is in days.  This offset is read into \verb+boundaries.c+ and
converted to seconds with the line
{\footnotesize
\begin{verbatim}
toffSet = MPI_GetValue(DATAFILE,"toffSet","BoundaryVelocities",myproc)*secondsPerDay;
\end{verbatim}
}

\subsubsection{Running the example}

This example is run by typing \verb+make test+ in the \verb+suntans/main/examples/tides+ directory.
This will compile the boundary conditions specified in \verb+boundaries.c+
and link them with the main \verb+sun+ executable, and then run the test.  Data is copied from
the \verb+rundata+ directory to the \verb+data+ directory, which becomes the working directory.  Typing
\verb+make test+ again only copies \verb+rundata/suntans.dat+ to \verb+data/suntans.dat+ and runs
the example again.  Therefore, if any changes that are made to the grid or any associated parameters,
the directory should be flushed with \verb+make clobber+.  This ensures that typing \verb+make test+ will
recompute the grid.   

Upon running the tidal example, the simulation will
complain that the tidal input files (as specified by \verb+TideInput+ in \verb+suntans.dat+)
are not present, with the error message
\begin{verbatim}
Error opening data/tidecomponents.dat.0!
Writing x-y boundary locations to data/tidexy.dat.0 instead.
Error opening data/tidecomponents.dat.1!
Writing x-y boundary locations to data/tidexy.dat.1 instead.
\end{verbatim}
In order to create these files (for details see Section \ref{sec:tides_bc}), run the matlab script
\verb+tides.m+.  This script will read in the locations of the boundary points
in the \verb+tidexy.dat.*+ files and output the required \verb+tidecomponents.dat.*+ files.  In order
to run this script, the paths to the required m-files must be specified in the file
\verb+setup_tides.m+. These are given by
\begin{verbatim}
addpath ../../../mfiles
addpath ../../../mfiles/suntides
addpath /home/fringer/research/SUNTANS/tides/m_map
\end{verbatim}
The first two paths should not need to be changed since these represent the default locations of
the suntans mfiles and suntides directories.  However, the user must set the location of the path to
the m\_map directory, which contains routines to convert from lon/lat coordinates to Cartesian coordinates
using the UTM projection.  This package can be downloaded from\\
\verb+http://www.eos.ubc.ca/~rich/map.html+
Note that the data directory in the \verb+suntides+ directory must also be unpacked in order to run
the \verb+suntides.m+ script.  This can be done by typing \verb+make+ in the \verb+suntans/mfiles/suntides+ directory.
Running the \verb+tides.m+ script will generate the \verb+tidecomponents.dat+ files
and will display the following message:
\begin{verbatim}
File ./data/tidexy.dat.0, found 40 boundary points.
File ./data/tidexy.dat.1, found 32 boundary points.
\end{verbatim}
This indicates that the locations of the boundary edges were found in the files \verb+tidexy.dat.*+ and these
were output into the \verb+tidecomponents.dat.*+ files in the \verb+data+ directory.
Once these files have been created, the example can be run with \verb+make test+, which will proceed as
follows:
\begin{verbatim}
Running suntans...
Processor 0,  Total memory: 1.77 Mb, 3865 cells
All processors: 3.58 Mb, 7719 cells (474 bytes/cell)
Outputting data at step 1 of 13440
5% Complete. 5.11e-02 s/step; 652.11 s remaining.
Outputting data at step 1344 of 13440
10% Complete. 4.85e-02 s/step; 587.19 s remaining.
15% Complete. 4.68e-02 s/step; 534.33 s remaining.
Outputting data at step 2688 of 13440
.
.
.
\end{verbatim}
This indicates that the simulation (which by default runs on two processors) takes 0.05 seconds per
time step to run, which is 1800 times faster than real time (since the time step, as specified by
the \verb+dt+ variable in \verb+suntans.dat+ is 90 s).  Data is output every 1344 time steps as
indicated by the \verb+ntout+ variable in \verb+suntans.dat+.  The simulation will run for a total
of \verb+nsteps=13440+ time steps or a total simulation time of \verb+nsteps*dt+=14 days.  Note
that the values of the horizontal (\verb+nu_H+) and vertical (\verb+nu+) eddy-viscosities (called ``laminar'' because they
are constant) are large in order to prevent the buildup of grid-scale energy over the course of
the simulation on a relatively coarse grid.

Output of the results can be viewed with the \verb+compare_to_otis.m+ matlab script
(which requires the correct path locations in \verb+setup_tides.m+).
This script compares the results of SUNTANS to the OTIS results at the location specified in the file
\verb+data/dataxy.dat+ (in UTM coordinates).  For details on how to specify these points
and analyze data from these points, see the header in the file \verb+suntans/main/profiles.c+ and the
associated mfile \verb+suntans/mfiles/profplot.m+.  The associated
lines in the present example in \verb+suntans.dat+ are given by
\begin{verbatim}
########################################################################
#
# For output of data
#
########################################################################
ProfileVariables        hu    # Only output free surface and currents
DataLocations   dataxy.dat    # dataxy.dat contains column x-y data
ProfileDataFile profdata.dat  # Information about profiles is in profdata.dat
ntoutProfs              1     # Output profile data every 1 time step
NkmaxProfs              1     # Only output the top 1 z-level
numInterpPoints         1     # Output data at the nearest neighbor.
\end{verbatim}
These indicate that free surface and velocity data at the locations specified in the file \verb+dataxy.dat+
will be output every time step (\verb+ntoutProfs+).  Because \verb+NkmaxProfs+ is set to 1, only the
top cell is output.  Otherwise, if \verb+NkmaxProfs+ is zero, all layers are output (this example has
\verb+Nkmax=10+ layers).  The \verb+numInterpPoints+ variable indicates how many of the points nearest to
the desired point are output.  SUNTANS does not interpolate output data; this is left up to the user.
In the present example it suffices to output nearest-neighbor data.  The output data can be compared to
the predictions of OTIS with the script \verb+compare_to_otis.m+.  Over the 14-day
period, the results are given in Figure \ref{fig:compare_to_otis_default}.  It should be noted that because
the OTIS data used for this example is quite coarse (1 degree) and because the interpolation used to
obtain the constituents near the coastline can lead to inaccurate predictions of the tidal boundary conditions
in the shallow areas of the domain, this comparison should not be regarded as a measure of the accuracy
of SUNTANS tidal simulations.  The user is encouraged to obtain higher-resolution tidal data for more accurate
simulations.  For more detail please visit Brian Dushaw's web page\footnote{
http://909ers.apl.washington.edu/~dushaw/tidegui/tidegui.html}.
To highlight the differences associated with the coarse tidal data, Figure \ref{fig:compare_interp} compares
the results of SUNTANS when employing three different interpolation techniques to obtain the tidal data
at the SUNTANS boundaries.  The particular interpolation method can be changed in the function \verb+suntans/mfiles/suntides/get_tides.m+,
and the results are strikingly different because the interpolation method has a significant effect on
forcing values near the coastline, since it is assumed that land values from OTIS are identically zero.
\insertfig{0.75}{figures/compare_to_otis_default}{Comparison of SUNTANS to OTIS at the location specified
in the dataxy.dat file.}{fig:compare_to_otis_default}
\insertfig{0.75}{figures/compare_interp}{Comparison of SUNTANS results when using 
different matlab interpolation techniques to interpolate
the tidal constituents from OTIS onto the SUNTANS boundary points. Legend: red=linear 
(same as Figure \ref{fig:compare_to_otis_default}), blue=cubic, black=spline.}
{fig:compare_interp}


\subsection{Time accuracy} \label{sec:timeaccuracy}

This example is in the \verb+examples/accuracy+ directory and it
serves to verify the time accuracy of SUNTANS.  Details of the time accuracy
of SUNTANS and this particular test case can be found in Fringer\etal~\cite{FRINGER[2005]}.
The parameter file is given by
\verb+suntans.in+, which lacks the parameters \verb+dt+ and \verb+nsteps+.  These
are copied to the actual parameter file \verb+data/suntans.dat+ upon running the
script \verb+accuracy.sh+, which runs SUNTANS with different time steps and computes
accuracy criteria using the \verb+accuracy.c+ code.  This code is a good example
of how to read in and analyze SUNTANS data using the C programming language.  More
details are given below.

\subsubsection{Grid}

The domain is 100 m long and 100 m deep, and it
uses 100 vertical levels and 100 equilateral triangles, as defined in the planar
straight line graph file \verb+rundata/oned.dat+, which was created with the
\verb+onedgrid.m+ m-file which can be downloaded from 

\medskip
\noindent
\mfiledownload.

\medskip
\noindent
All boundary edges are of type 1 (closed) boundaries.
The depth is constant and is defined in \verb+initialization.c+ in the 
\verb+ReturnDepth+ function with the line
\begin{verbatim}
return 100;
\end{verbatim}
Note that in order for this depth to be specified, the \verb+IntDepth+ parameter must
be set to 0 in the \verb+suntans.dat+ parameter file.

\subsubsection{Initial conditions}

The initial velocity field is quiescent, the initial free surface height is 0,
and the initial density distribution is defined in the function \verb+ReturnSalinity+
in \verb+initialization.c+ as
\begin{verbatim}
  return -.03*tanh(2.0*2.6467/20*(z+50-cos(PI*x/100)));
\end{verbatim}
Because $\beta=1$ in \verb+suntans.dat+ (the parameter \verb+beta+), this is also the initial density 
distribution, since $(\rho-\rho_0)/\rho_0 = \beta(s-s_0)$ More generally, then, 
this density distribution is given by
\[
\frac{\rho}{\rho_0} = -\frac{1}{2}\frac{\Delta\rho}{\rho_0}
\tanh\left[\frac{2\mbox{tanh}^{-1}(\alpha)}{\delta}\left(z+\frac{D}{2}-\zeta\right)\right]\,,
\]
where
\begin{itemize}
\item $\Delta\rho/\rho_0=0.06$: Density difference between layers.
\item $\alpha=0.99$: Parameter that defines interface extent.
\item $\delta=20$ m: Interface thickness.
\item $z$: Vertical coordinate
\item $D=100$ m: Depth
\item $\zeta = \cos(\pi x/L)$: Interface profile, where $L=100$ m is the domain length.
\end{itemize}
and is shown in Figure \ref{fig:initdens}. 
\insertfig{1}{figures/initdens}{Initial density distribution in $x$-$z$ (a) and as a profile at $x=L/2$ (b)
for the time accuracy test case.}{fig:initdens}

\subsubsection{Boundary conditions}

Because all boundaries of the planar
straight line graph (\verb+rundata/oned.dat+) are of type 1, the functions in the \verb+boundaries.c+ file are
not used in this example.

\subsubsection{Running the test}

The accuracy test case is run at the command line with
\begin{verbatim}
make test
\end{verbatim}
This will compile the local initial and boundary condition files and link them with the
executable \verb+../../sun+.  The script \verb+accuracy.sh+ loops through different
time step sizes given by $\Delta t_n = \Delta t_0/2^n$, with $n=0,1,\dots,5$, where
$\Delta t_0 = 0.1$ s.  If the reference solution is given by the solution $\phi^{ref}$
with a time step size of $\Delta t_0/32$, then the error between a solution using
$\Delta t=\Delta t_0/2^n$ and the 
reference solution can be calculated with
\[
Error(n)^2 = \frac{\sum_{i=0}^{Nc-1}\sum_{k=0}^{N_{kmax}-1} \left(\phi_{i,k}-\phi^{ref}_{i,k}\right)^2}
{\sum_{i=0}^{Nc-1}\sum_{k=0}^{N_{kmax}-1} \left(\phi^{ref}_{i,k}\right)^2}\,.
\]
Since this is a second-order method, we must have $Error(n)/Error(n-1)=4$ since the time step size
for $Error(n)$ is double that of $Error(n-1)$.  Therefore, $Error(n)/Error(0) = 4^n = (\Delta t_0/\Delta t_n)^2$,
$n=0,1,\dots,4$.  After
the \verb+accuracy.sh+ script finishes, the results are analyzed with the code in \verb+accuracy.c+,
which computes $Error(n)$.  The output of this code is given by
\begin{verbatim}
Error results (Error(n)/Error(0)):

Your results:
--------------------------------------------------
dt0/dt  U       W       S       Q       Q0      h
--------------------------------------------------
1       1.0     1.0     1.0     1.0     1.0     1.0
2       3.9     3.9     4.0     4.1     1.2     4.0
4       15.6    15.5    16.2    16.4    2.1     16.4
8       65.1    64.8    68.1    68.9    4.4     69.1
16      324.7   323.5   340.6   344.8   12.8    345.9

Reference results:
1       1.0     1.0     1.0     1.0     1.0     1.0
2       3.9     3.9     4.0     4.1     1.2     4.0
4       15.6    15.5    16.2    16.4    2.1     16.4
8       65.1    64.8    68.1    68.9    4.4     69.1
16      324.7   323.5   340.6   344.8   12.8    345.9

Difference (relative):
1       0.00    0.00    0.00    0.00    0.00    0.00
2       -0.00   0.00    0.00    0.00    0.00    -0.00
4       -0.00   0.00    0.00    0.00    0.00    -0.00
8       -0.00   0.00    0.00    0.00    0.00    -0.00
16      -0.00   0.00    0.00    0.00    0.00    -0.00
\end{verbatim}
The code outputs the results from the current run, then displays the results that
have been saved from a working version of this simulation, and then plots the relative
difference between your results and the stored results as
\[
\mbox{Difference} = \frac{\mbox{Your($Error(n)/Error(0)$)} - \mbox{Stored($Error(n)/Error(0)$)}}
{\mbox{Stored($Error(n)/Error(0)$)}}\,.
\]
Each column represents:
\begin{itemize}
\item \verb+dt0/dt+: $\Delta t_0/\Delta t^n$, $n=0,1,\dots 4$
\item \verb+U+: Horizontal velocity.
\item \verb+W+: Vertical velocity
\item \verb+S+: Salinity (or density, since $\beta=1$).
\item \verb+Q+: Nonhydrostatic pressure using the second-order Adams-Bashforth extrapolation
using the last two time steps (for details see Fringer\etal~\cite{FRINGER[2005]}).
\item \verb+Q0+: Nonhydrostatic pressure without the extrapolation of \verb+Q+.
\item \verb+h+: Free surface.
\end{itemize}
The first time you download and run the examples, the results
you obtain should be identical to the stored results since they were obtained with the
same code, and the relative difference should be identically zero.  Changing parts of the
code may or may not affect the accuracy, but if it does then this will be reflected by
nonzero values for the relative difference.

\subsection{Lock exchange} \label{sec:lockexchange}

This example is in the \verb+examples/lockexchange+ directory and it 
demonstrates the nonhydrostatic internal lock exchange problem.  Due to the
coarseness of the grid being used, numerical diffusion acts to limit the formation of
the KH billows along the interface.  These billows can easily be obtained by increasing
the resolution, following the simulation outlined by Fringer\etal~\cite{FRINGER[2005]}.

\subsubsection{Grid}

The grid in this example is obtained with the 
\verb+onedgrid.m+ m-file which can be downloaded from 

\medskip
\noindent
\mfiledownload.

\medskip
\noindent
In this case, the grid contains 200 cells in the horizontal and 20 in the vertical, with
a length of 100 m and a depth of 20 m.  This constant depth is specified in \verb+initialization.c+
in the function \verb+ReturnDepth+ with the line
\begin{verbatim}
return 20;
\end{verbatim}
Note that in order for this depth to be specified, the \verb+IntDepth+ parameter must
be set to 0 in the \verb+suntans.dat+ parameter file.  As shown in Figure \ref{fig:stretched},
this grid is stretched in the
vertical in order to provide extra resolution at the bottom boundary.  This is done by
specifying a negative stretching factor of $r=-1.025$ (\verb+suntans.dat: rstretch = -1.025+),
which causes each grid layer to be 1.025 times thicker than the layer beneath it.  
\insertfig{0.8}{figures/stretched}{Depiction of the vertically stretched grid for the lock exchange
problem.  Every fourth vertical cell face plotted for clarity.}{fig:stretched}

\subsubsection{Initial conditions}

The initial velocity field is quiescent, the initial free surface height is 0,
and the initial density distribution is defined in the function \verb+ReturnSalinity+
in \verb+initialization.c+ as
\begin{verbatim}
  return -.001*tanh(2.0*2.6467/5*(x-50.0));
\end{verbatim}
Because $\beta=1$ in \verb+suntans.dat+ (the parameter \verb+beta+), this is also the initial density 
distribution, since $(\rho-\rho_0)/\rho_0 = \beta(s-s_0)$ More generally, then, 
this density distribution is given by
\[
\frac{\rho}{\rho_0} = -\frac{1}{2}\frac{\Delta\rho}{\rho_0}
\tanh\left[\frac{2\mbox{tanh}^{-1}(\alpha)}{\delta}\left(x-\frac{L}{2}\right)\right]\,,
\]
where
\begin{itemize}
\item $\Delta\rho/\rho_0=2\times 10^{-3}$: Density difference.
\item $\alpha=0.99$: Parameter that defines interface extent.
\item $\delta=5$ m: Interface thickness.
\item $x$: Horizontal coordinate.
\item $L=100$ m: Length
\end{itemize}
and is shown in Figure \ref{fig:initlock}. 
\insertfig{0.8}{figures/initlock}{Initial density distribution in $x$-$z$ (a) and as a profile at $x=L/2$ (b)
for the lock exchange test case.}{fig:initlock}

\subsubsection{Boundary conditions}

Because all boundaries of the planar
straight line graph (\verb+rundata/oned.dat+) are of type 1, the functions in the \verb+boundaries.c+ file are
not used in this example.

\subsubsection{Running the test}

This test case is run with the command
\begin{verbatim}
make test
\end{verbatim}
The simulation runs for 500 time steps (\verb+suntans.dat: nsteps = 500+) with a time step size of 0.2 s (\verb+suntans.dat: dt = 0.2+).
For this grid, given that the Voronoi distance for the equilateral triangles with sides of length $\Delta x=L/200$
is $D_g = \Delta x/\sqrt{3} = 0.2887$ m, and the maximum velocity is roughly $u_{max}=0.5$ m s$^{-1}$ (it actually
never exceeds $0.44$ m s$^{-1}$),
the horizontal Courant number is $C_x=u_{max}\Delta t/D_g = 0.35$, and the vertical Courant number is
$C_z=w_{max}\Delta t/\Delta z_{min}=0.2\mbox{m s}^{-1} 0.2\mbox{s}/0.78 m=0.05$.  Because central differencing
is employed for advection of momentum (\verb+suntans.dat: nonlinear = 2+), 
the horizontal grid Peclet number must satisfy the one-dimensional stability criterion which requires
that  $Pe_{\Delta x}<2/C_x$.  This is an approximation and is not as restrictive as it should be for multidimensional,
unstructured-grid problems, but it serves as a good approximation (See Fletcher~\cite{FLETCHER[1997]} for details).
For the horizontal stability, then, this requires that $\nu_H\ge u_{max}^2\Delta t/2 = 0.025$ m$^2$ s$^{-1}$ (\verb+suntans.dat: nuH = 0.025+).
Likewise, for vertical stability, $\nu = 0.016$  m$^2$ s$^{-1}$ (\verb+suntans.dat: nu = 0.016+).

The results can be viewed with the \verb+sunplot+ gui from the main source directory with
\begin{verbatim}
./sunplot --datadir=examples/lockexchange/data
\end{verbatim}
This will open up a planview of the one-dimensional grid of equilateral triangles, showing that
there are 26 (1+\verb+nsteps+/\verb+ntout+) time steps to plot.  To view the
profile plot, depress the \button{Profile} button with the middle mouse button.  The last time step
of the velocity vectors along with the salinity field can be viewed with the following buttons:
\begin{enumerate}
\item Right mouse on \button{$-->$}:  To get to last time step.
\item Left mouse on \button{Vectors}: Turn on velocity vectors.
\item 3$\times$ Right mouse on \button{Vectors}: Increase vector length by a factor of 8.
\item 2$\times$ Left mouse on \button{$>$}: Increase \verb+iskip+ to 3 to plot every third vector for clarity.
\end{enumerate}
The display should appear as it does in Figure \ref{fig:lock}.
\insertfig{0.6}{figures/lock}{Sunplot display of the lock exchange example after 500 time steps.}{fig:lock}

\subsection{Boundary condition example} \label{sec:boundary_ex}

This example is in the \verb+examples/boundaries+ directory and it simulates a simplified
river plume in order to demonstrate the use of velocity as well as scalar boundary conditions
at the inflow and the outflow.

\subsubsection{Grid}

The two-dimensional grid of equilateral triangles is created with the m-file \verb+twodgrid.m+,
which can be downloaded from

\medskip
\noindent
\mfiledownload.

\medskip
\noindent
As shown in Figure \ref{fig:plumegrid}, it is 3 km long by 1 km wide with a total of 1215 equilateral triangles
with edge lengths of 75 m.  The eastern and southern boundaries
are marked as type 2 edges.  These will be specified in \verb+boundaries.c+.  The depth is
10 m and is specified in \verb+initialization.c+ as
\begin{verbatim}
return 10;
\end{verbatim}
There are 10 vertical levels (\verb+suntans.dat: Nkmax = 10+).
\insertfig{.8}{figures/plumegrid}{Two-dimensional planview of the river plume grid, showing the
boundary edges of type 2 in bold.}{fig:plumegrid}

\subsubsection{Initial conditions}

As specified in \verb+initialization.c+, the initial conditions are straightforward in
that all quantities are initialized to zero except the temperature field, which is set to 1
everywhere.  It is treated as a passive
tracer when $\gamma=1$ (\verb+suntans.dat: gamma = 1+).

\subsubsection{Boundary conditions}

\begin{itemize}
\item[] {\bf Eastern Boundary: open}\\
The eastern boundary fluxes normal to the faces are specified while values used for
advection of momentum and scalars are upwinded
\begin{itemize}
\item[] {\bf Boundary fluxes}\\
As specified in \verb+boundaries.c+, the eastern boundary is an open boundary and employs the
linearized boundary condition on the velocity with
\[ u_b = -h_b\sqrt{\frac{g}{d}} \,,.\]
where $h_b$ is the free-surface height, $g$ is gravitational acceleration, and $d$ is
the depth, and the minus sign implies flux out of the domain when $h_b>0$.
This is enforced in the \verb+OpenBoundaryFluxes+ function with the lines
\begin{verbatim}
if(grid->yv[ib]>50) {
   for(k=grid->etop[j];k<grid->Nke[j];k++) 
     ub[j][k]=-phys->h[ib]*sqrt(prop->grav/grid->dv[ib]);
...
\end{verbatim}
The \verb+ib+ index is the index of the cell adjacent to the boundary, and the if-statement
is required to distinguish this eastern outflow boundary from the southern boundary.  Note
that because the edge lengths are $\Delta x = 75$ m, this implies that the distance between the Voronoi points
and the boundary edges is $\Delta x/(2\sqrt{3})=21.7$ m.  Therefore, we know that if the location
of the boundary Voronoi point \verb+ib+ adjacent to a boundary edge \verb+j+ 
is greater than $50$ m, it must be the eastern boundary.
\item[] {\bf Boundary velocities}\\
At the eastern open boundary, the boundary velocities which are used for advection of momentum
out of the domain are given by the upwind velocity components.  This is specified in the
\verb+BoundaryVelocities+ function with the lines
\begin{verbatim}
if(grid->yv[ib]>50) {
   for(k=grid->etop[j];k<grid->Nke[j];k++) {
      phys->boundary_u[jptr-grid->edgedist[2]][k]=
                                  phys->uc[ib][k];
      phys->boundary_v[jptr-grid->edgedist[2]][k]=
                                  phys->vc[ib][k];
      phys->boundary_w[jptr-grid->edgedist[2]][k]=
                             0.5*(phys->w[ib][k]+phys->w[ib][k+1]);
   } 
\end{verbatim}
Because the flux of momentum is calculated at the vertical centers of the vertical faces,
the upwind vertical velocity is interpolated from the upper (k) and lower (k+1) faces of the kth cell.
\item[] {\bf Boundary scalars}\\
The outflow conditions on the scalars are imposed by specified the upwind scalar quantity
at the outflow face.  This is set in the function \verb+BoundaryScalars+ with the lines
\begin{verbatim}
for(k=grid->ctop[ib];k<grid->Nk[ib];k++) {
   phys->boundary_T[jptr-grid->edgedist[2]][k]=phys->T[ib][k];
   phys->boundary_s[jptr-grid->edgedist[2]][k]=phys->s[ib][k];
}
\end{verbatim}
\end{itemize}
\end{itemize}

\begin{itemize}
\item[] {\bf Southern boundary: specified}\\
All quantities at the southern boundary are specified.  Since this example simulates a river
inflow, the velocity and scalar fields are specified along only part of the southern boundary, while
the rest of the boundary is kept at zero inflow to represent a solid boundary.  This requires if-statements to
determine which of the southern boundary edges fall between the extent of the inflowing river.
An alternative is to set the markers of the southern boundary edges that fall within the boundaries
of the river to 2, and the rest to 1. 
\begin{itemize}
\item[] {\bf Boundary fluxes}
Because the velocity components at the southern boundary are specified, the incoming boundary
flux is computed in \verb+OpenBoundaryFluxes+ using the velocity components specified in the \verb+BoundaryVelocities+ function,
but only within the extent of the 300 m wide river, which is defined for $900 < x < 1200$ m (4 cell faces), as in:
\begin{verbatim}
if(grid->xv[ib]>900&&grid->xv[ib]<1200)
  for(k=grid->etop[j];k<grid->Nke[j];k++) 
    ub[j][k]=
      phys->boundary_u[jptr-grid->edgedist[2]][k]*grid->n1[j]+
      phys->boundary_v[jptr-grid->edgedist[2]][k]*grid->n2[j];
\end{verbatim}
Since \verb+boundary_u+ and \verb+boundary_v+ store the inflow velocity components, then this
is setting the flux at the inflow to the normal component of the velocity at the inflow, i.e.
\[
U_{inflow} = {\mathbf u}_{inflow}\cdot {\mathbf n}_{inflow}\,,
\]
where ${\mathbf n}_{inflow}$ is, by convention, the inward pointing normal at the boundary face.
\item[] {\bf Boundary velocities}
The north-south velocity is specified at the inflow for $900 < x < 1200$ over a specified depth,
which represents the river inflow, such that
\begin{eqnarray}
u_{inflow} &=& 0\,,\nonumber\\
v_{inflow} &=& \left\{\begin{array}{ll}
                    \mbox{amp} & z>-D_r \\
		    0 & \mbox{otherwise\,,}
		    \end{array}
             \right.\nonumber\\
w_{inflow} &=& 0\,.\nonumber
\end{eqnarray}
where amp is specified in \verb+suntans.dat+ as $0.01$ m s$^{-1}$, and
$D_r=3$ m is the inflow depth.  In \verb+boundaries.c+, this inflow is implemented with
\begin{verbatim}
if(grid->xv[ib]>900 && grid->xv[ib]<1200) {
  z=0;
  for(k=grid->etop[j];k<grid->Nke[j];k++) {
    z-=0.5*grid->dzz[ib][k];
    if(z>-3.0)
      phys->boundary_v[jptr-grid->edgedist[2]][k]=prop->amp;
    z-=0.5*grid->dzz[ib][k];
  }
}
\end{verbatim}
All other components are set to zero at the beginning of the main loop in the function \verb+BoundaryVelocities+
with
\begin{verbatim}
for(k=grid->etop[j];k<grid->Nke[j];k++) {
  phys->boundary_u[jptr-grid->edgedist[2]][k]=0;
  phys->boundary_v[jptr-grid->edgedist[2]][k]=0;
  phys->boundary_w[jptr-grid->edgedist[2]][k]=0;
}
\end{verbatim}
\item[] {\bf Boundary scalars}
The inflow boundary condition for the salinity (or density, when $\beta=1$) field
represents an inflow of fresh water with a density anomaly of $\Delta \rho/\rho_0=-10^{-4}$, with a surface depth of $D_r=3$ m,
such that
\[
\rho_{inflow} = \left\{\begin{array}{ll}
                    \Delta \rho & z>-D_r \\
		    0 & \mbox{otherwise\,.}
		    \end{array}
             \right.
\]
The inflow boundary condition for temperature is a no-gradient condition, and as a result the temperature
just outside the boundary is equal to that just inside the boundary.  These boundary conditions for salinity (density)
and temperature are given in \verb+boundaries.c+ as
\begin{verbatim}
if(grid->yv[ib]<50)
  if(grid->xv[ib]>1000 && grid->xv[ib]<1200)
    for(k=grid->ctop[ib];k<grid->Nk[ib];k++) {
      z-=0.5*grid->dzz[ib][k];
      if(z>-3.0) {
        phys->boundary_T[jptr-grid->edgedist[2]][k]=phys->T[ib][k];
        phys->boundary_s[jptr-grid->edgedist[2]][k]=-0.0001;
      } else {
        phys->boundary_T[jptr-grid->edgedist[2]][k]=phys->T[ib][k];
        phys->boundary_s[jptr-grid->edgedist[2]][k]=0;
      }
      z-=0.5*grid->dzz[ib][k];
    }
\end{verbatim}
\end{itemize}
\end{itemize}

\subsubsection{Running the test}

Upon typing \verb+make test+, the simulation runs for a total of 3000 time steps (\verb+suntans.dat: nsteps = 3000+)
with a time step of 60 s (\verb+suntans.dat: dt = 60+), outputting
data every 120 time steps (\verb+suntans.dat: ntout = 120+).  While the data is running you can view the
results from the main source directory with
\begin{verbatim}
./sunplot --datadir=./examples/boundaries/data
\end{verbatim}
As can be seen from the results, the river plume forms a bulge as well as a coastal current resulting from
the nonzero Coriolis parameter of $f=5\times 10^{-4}$ (\verb+suntans.dat: Coriolis_f = 5e-4+).
After 3000 time steps, the density anomaly at the upper layer is depicted in Figure \ref{fig:plume}.  This
figure was obtained using the m-file \verb+sunsurf.m+, which can be downloaded from

\medskip
\noindent
\mfiledownload.

\medskip
\noindent
If you are running this m-file from the \verb+examples/boundaries+ directory, then the plot in Figure \ref{fig:plume}
was obtained with the command
\begin{verbatim}
timestep = 26;
klevel = 1;
processor = 0;
sunsurf('s','.../data',timestep,klevel,processor);
\end{verbatim}
Note that \verb+sunsurf.m+ requires \verb+unsurf.m+. 
\insertfig{0.8}{figures/plume}{River plume example showing the salinity (density) field after 3000 time steps.}{fig:plume}

\subsection{Cavity flow} \label{sec:cavityflow}

The cavity flow example (\verb+examples/cavity+) provides a means to 
test advection of momentum, no-slip boundary conditions, and  rigid lid 
conditions.  There are two examples, \verb+testXY+ and \verb+testXZ+ which
comprise a cavity flow simulation in the XY and XZ planes, respectively. 
The \verb+testXY+ example demonstrates use of a stretched grid and no-slip 
boundary conditions.  The \verb+testXZ+ example shows usage of the no-slip 
boundary condition as a means to force flows within the domain and highlights 
the utility of a rigid lid computation.  These cases provide a starting point 
for examination of the effect of numerical methods employed on balances 
between the nonlinear advection of momentum, the pressure gradient, 
and viscous dissipation.

\subsubsection{Grid}

The grid for the \verb+testXY+ case was generated with Rusty Holleman's 
TOM gridding solver with $\Delta x\approx 0.01$ on the outer boundary and 
$\Delta x \approx 0.1$ on the inner part of the domain for a total 
of 4016 cells with a single layer (\verb+suntans.dat: Nkmax = 1+). 
The \verb+testXZ+ grid
was derived from a 1D row of equilateral grid cells with right-angle triangles 
included on both sides so that the enforced boundary condition would be normal 
to the flow from the prescribed driven lid in the z-direction on the left side 
of the domain. There are 128 cells in the horizontal and 128 cells in the 
vertical (\verb+suntans.dat: Nkmax = 128+). Note that due to the coarseness of the grids, there is some 
deviation from  the results in Zang\etal~\cite{Zang[1994]}.  

\subsubsection{Initial conditions}

Flows within the domain are intially quiescent with depths of 1.0 and free 
surface heights of 0.  The simulation is for pure water with constant density
and viscosity.

\subsubsection{Boundary conditions}

Boundary conditions for the \verb+testXY+ case are specified by ensuring that 
all boundary markers are no-slip boundary conditions of type \verb+4+.  The 
\verb+testXZ+ case is somewhat more complicated, as the western and eastern-most
edge is of type \verb+4+ with the channel sides parallel to the flow closed with
type \verb+1+.  No-slip top and bottom boundary conditions are specified by
setting \verb+CdT -1+ and \verb+CdB -1+ in \verb+suntans.dat+.  Deflection of
the free surface, as would be expected to be developed in response to flows, is
controlled via the rigid lid approximation by setting \verb+grav=9.81e6+ in 
\verb+suntans.dat+.

For the \verb+testXY+ case all boundary velocities are 0 except the lid
as specified in BoundaryVelocities() of \verb+boundariesXY.c+ via
\begin{verbatim}
  for(jptr=grid->edgedist[4];jptr<grid->edgedist[5];jptr++) {
    j = grid->edgep[jptr];
    ib=grid->grad[2*j];
    boundary_index = jptr-grid->edgedist[2];

    for(k=grid->ctop[ib];k<grid->Nk[ib];k++) {
      if(grid->ye[j]<0.000010)
        phys->boundary_u[boundary_index][k]= 1.0;
      else
       phys->boundary_u[boundary_index][k] = 0.0;
       phys->boundary_v[boundary_index][k] = 0.0;
       phys->boundary_w[boundary_index][k] = 0.0;
    }
  }
\end{verbatim}
Observe that this function specifies the boundary value for each
edge of type \verb+4+, where the lid (located at the southern end of
the domain) is driven by a constant flow towards the east.   

The boundary conditions for \verb+testXZ+ are somewhat similar, but have the
driven lid on the western most end of the domain with flow towards the 
south.  Their specification in BoundaryVelocities() in \verb+boundariesXZ.c+
follows.
\begin{verbatim}
  for(jptr=grid->edgedist[4];jptr<grid->edgedist[5];jptr++) {

    j = grid->edgep[jptr];
    ib=grid->grad[2*j];
    boundary_index = jptr-grid->edgedist[2];

    for(k=grid->ctop[ib];k<grid->Nk[ib];k++) {
      if(grid->xe[j]<0.5)
        phys->boundary_w[boundary_index][k]= -1.0;
      else
        phys->boundary_w[boundary_index][k]=0.0;
      phys->boundary_v[boundary_index][k] = 0.0;
      phys->boundary_u[boundary_index][k] = 0.0;
    }
  }
\end{verbatim}
The lid velocity for both cases is unity to allow easy specification of 
Reynolds numbers via $\nu$. 

\subsubsection{Running the test}
This test case is run with the command
\begin{verbatim}
make test
\end{verbatim}
which runs the \verb+testXY+ and \verb+testXZ+ cases in turn.  These cases can 
be individually run with the commands
\begin{verbatim}
make testXY
\end{verbatim}
and
\begin{verbatim}
make testXZ
\end{verbatim}
Central differencing for advection of momentum 
(\verb+suntans.dat: nonlinear = 2+) is employed and nonhydrostatic pressure
used for the \verb+testXZ+ case (\verb+suntans.dat: nonhydrostatic = 1+).  Small
time steps of \verb+0.001+ and \verb+0.005+ are employed to resolve 
eddies over a total time of \verb+40+, corresponding to a rough approximation
for steady state.  Viscosity controls the Reynolds number and is set to 
yield $Re=3200$ (\verb+suntans.dat: nu = 0.0003125+ and 
\verb+suntans.dat: nu_H = 0.0003125+) for comparisons with 
Zang\etal~\cite{Zang[1994]}.  Results from the simulation can be compared with
these literature data values via the Matlab scripts \verb+CompareResultXY.m+ and 
\verb+CompareResultXZ.m+.  A more resolved case than the examples (13,500 cells)
is shown in  Figure \ref{fig:cavity}, where $L=1$ is the square domain's length,
$U=1.0$ is lid velocity, and $u$ and $v$ are the zonal and meridional velocities,
respectively.  
\insertfig{0.8}{figures/cavity}{Velocity profiles for refined 
XY cavity flow (lines) compared with literature data from 
Zang\etal~\cite{Zang[1994]}(squares)}{fig:cavity}

\subsection{Internal waves} \label{sec:internalwaves}

This example is found in the \verb+examples/iwaves+ directory, and it demonstrates the formation
of internal wave beams at an idealized coastal shelf break in two dimensions $x$-$z$.  The barotropic
M2 tide is forced at the western boundary, forcing tidal currents over the break and thus generating
internal wave beams.

\subsubsection{Grid}

The grid in this example is obtained with the 
\verb+onedgrid.m+ m-file which can be downloaded from 

\medskip
\noindent
\mfiledownload.

\medskip
\noindent
In this case, the one-dimensional grid contains 100 cells in the horizontal and 100 in the vertical, and
the domain is 100 km long with a maximum depth of 3000 m.  All edges are of type 1 except for the western
boundary edge, which is of type 2.
The shelf slope geometry is given by
\[
d(x) = \left\{\begin{array}{ll}
D_0 & x\le x_{mid}-L_s/2\\
D_s & x> x_{mid}+L_s/2\\
D_0 - \left(D_0-D_s\right)\left[(x-x_{mid})/L_s+\frac{1}{2}\right] & \mbox{otherwise}\,.
\end{array}\right.
\]
where
\begin{itemize}
\item[] $L_s=20$ km: Horizontal extent of slope.
\item[] $x_{mid}=65$ km: Center of slope.
\item[] $D_0=3000$ m: Maximum depth.
\item[] $D_s=500$ m: Shelf depth.
\end{itemize}
This depth profile is specified in \verb+boundaries.c+ in the function \verb+ReturnDepth+ with
\begin{verbatim}
Ls = 20000;
xmid = 65000;
D0 = 3000;
Ds = 500;
if(x<=xmid-Ls/2)
  return D0;
else if(x>xmid+Ls/2)
  return Ds;
else
  return D0-(D0-Ds)*((x-xmid)/Ls+0.5);
\end{verbatim}
As with all the present examples, in order for this depth to be specified, the \verb+IntDepth+ parameter must
be set to 0 in the \verb+suntans.dat+ parameter file, otherwise, depth data must be supplied in the
file specified by \verb+depth+ in \verb+suntans.dat+.  As shown in Figure \ref{fig:iwstretched},
this grid is stretched in the
vertical in order to provide extra resolution at the top boundary.  This is done by
specifying a positive stretching factor (as opposed to a negative stretching factor for the
bottom boundary layer, as in Section \ref{sec:lockexchange}) of $r=1.025$ (\verb+suntans.dat: rstretch = 1.025+),
which causes each grid layer to be 1.025 times thicker than the layer {\bf above} it.  
\insertfig{0.8}{figures/iwstretched}{Depiction of the bathymetry and vertically stretched grid for the 
internal waves problem. Minimum grid spacing: 6.94 m, maximum: 79.94 m.}{fig:iwstretched}

\subsubsection{Initial conditions}

The initial conditions for the internal wave problem, as specified in \verb+initialization.c+,
are quiescent velocity field, zero free-surface, and constant temperature field (the passive
tracer) of $T=1$, with an initial salinity field (or density, since
\verb+beta=1+ in \verb+suntans.dat+) of
\[
s(z) = \left\{\begin{array}{ll}
\Delta s\left(-z\right)^{\alpha_s} & z<-D_{pycnocline}\,,\\
\Delta s\left(D_{pycnocline}\right)^{\alpha_s} & \mbox{otherwise}\,,
\end{array}\right.
\]
where $\Delta s = 0.024$, $\alpha_s = 0.0187$, and $D_{pycnocline} = 20$.  These
parameters have been chosen to create a typical U.S. west coast density field.
This salinity (density) profile is 
implemented in the function \verb+ReturnSalinity+ in \verb+initialization.c+ as
\begin{verbatim}
deltaS = 0.024;
alphaS = 0.0187;
D_pycnocline = 20;

if(z<-D_pycnocline)
  return deltaS*pow(-z,0.0187);
else
  return deltaS*pow(D_pycnocline,0.0187);
\end{verbatim}

\subsubsection{Boundary conditions}

Boundary conditions are gradient-free on the temperature and salinity fields at the western
boundary.  This is implemented in the function \verb+BoundaryScalars+ in the file \verb+boundaries.c+
as
\begin{verbatim}
for(k=grid->ctop[ib];k<grid->Nk[ib];k++) {
  phys->boundary_T[jptr-grid->edgedist[2]][k]=phys->T[ib][k];
  phys->boundary_s[jptr-grid->edgedist[2]][k]=phys->s[ib][k];
}
\end{verbatim}
where \verb+ib+ is the index of the Voronoi point adjacent to the boundary edge \verb+j+.

For the velocity field, the $x$-direction velocity is specified at the western boundary
as a sinusoidally varying barotropic velocity with a $M_2$ tidal period of 12.42 hours,
and the other components are zero.  This is implemented in the function \verb+BoundaryVelocities+
in the file \verb+boundaries.c+ as
\begin{verbatim}
for(k=grid->etop[j];k<grid->Nke[j];k++) {
  phys->boundary_u[jptr-grid->edgedist[2]][k]=
        prop->amp*sin(prop->omega*prop->rtime);
  phys->boundary_v[jptr-grid->edgedist[2]][k]=0;
  phys->boundary_w[jptr-grid->edgedist[2]][k]=0;
}
\end{verbatim}
The variables \verb+amp+ and \verb+omega+ are set in \verb+suntans.dat+ 
(\verb+amp = 4+ mm s$^{-1}$ and \verb+omega = 1.4026e-4+ rad s$^{-1}$),
and \verb+rtime+ is the simulation time, in seconds.  These velocities are
then used in the \verb+OpenBoundaryFluxes+ function to compute the velocity normal
to the boundary face with 
\begin{verbatim}
for(k=grid->etop[j];k<grid->Nke[j];k++) 
  ub[j][k]=phys->boundary_u[jptr-grid->edgedist[2]][k]*grid->n1[j]+
           phys->boundary_v[jptr-grid->edgedist[2]][k]*grid->n2[j];
\end{verbatim}
Since there is only one boundary, no if-statements are required to determine if
this is an open or closed boundary condition, such as in Section \ref{sec:boundary_ex}.
More complex partially-clamped boundary conditions can be implemented that allow the internal wave
field to exit the domain while still forcing the barotropic velocity field, but this is
beyond the scope of this simplified example.

\subsubsection{Running the test}

The test case is run at the command line with the command
\begin{verbatim}
make test
\end{verbatim}
which runs the simulation for 3000 time steps (\verb+suntans.dat: nsteps = 3000+)
with a time step size of 29.808 s (\verb+suntans.dat: dt = 29.808+) for a total of
2 $M_2$ tidal periods.  This test case uses the sponge layer which is hardwired into
SUNTANS.  The parameters are set to \verb+sponge_distance = 5000+ and 
\verb+sponge_decay = 7200+ (for details see Section \ref{sec:params}).

The results can be viewed with the \verb+sunplot+ gui from the main source directory with
\begin{verbatim}
./sunplot --datadir=examples/iwaves/data
\end{verbatim}
This brings up a planview of the one-dimensional grid of equilateral triangles.  To display
the profile plot of the results, press the \button{Profile} button with the middle mouse
button and then the \button{Axis} button with the left mouse button so that the axes fill the plot window.
This will display the salinity (density) field at the first time step, as shown in Figure 
\ref{fig:iwaves1}.  Plotting the u-velocity at data steps 11 and 21 will display the horizontal
velocity field after one and two periods of forcing, respectively, as shown in Figures 
\ref{fig:iwaves2} and \ref{fig:iwaves3}.  Note that even with the use of the sponge layer,
some internal wave energy is reflected back from the left boundary and affects the internal
wave field shown in Figure \ref{fig:iwaves3}.  In addition to using \verb+sunplot+, printouts for
these  plots can be obtained with the 
\verb+plotslice.m+ m-file which can be downloaded from 

\medskip
\noindent
\mfiledownload.

\medskip
\noindent
\insertfig{0.75}{figures/iwaves1}{The initial salinity (density) field for the internal wave 
test case.}{fig:iwaves1}
\insertfig{0.75}{figures/iwaves2}{The horizontal velocity field  (mm s$^{-1}$) after one $M_2$ tidal period for
the internal wave test case.}{fig:iwaves2}
\insertfig{0.75}{figures/iwaves3}{The horizontal velocity field (mm s$^{-1}$) after two $M_2$ tidal periods for
the internal wave test case.}{fig:iwaves3}

%\subsection{Wetting and drying (bug release!)} \label{sec:wettinganddrying}

%\subsubsection{Grid}
%\subsubsection{Initial conditions}
%\subsubsection{Boundary conditions}
%\subsubsection{Running the test}
