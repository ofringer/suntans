\section{Restarting Runs} \label{sec:restart}

It is always a good idea to design production-scale runs so that at any point in time if your
job crashes you can restart it without having lost too much data.  At the end of a run, SUNTANS
outputs the restart file specified by \verb+StoreFile+ in \verb+suntans.dat+.  To restart a run using
this restart file, you need to copy this file into the file specified by \verb+StartFile+ in \verb+suntans.dat+.
This reads the data required to continue a run.  For example, at the end of a two-processor run in which
the \verb+StoreFile+ variable is set to \verb+store.dat+, two restart files are created for each processor, 
namely \verb+store.dat.0+ and \verb+store.dat.1+.  To restart this run, set the entry for \verb+StartFile+ in
\verb+suntans.dat+ to \verb+start.dat+ and copy \verb+StoreFile+ to \verb+StartFile+ for each processor, i.e.
copy the contents of \verb+store.dat.0+ to \verb+start.dat.0+ and \verb+store.dat.1+ to \verb+start.dat.1+.

Since the new output data from the restart run will overwrite any existing data from a previous run, it is
a good idea to create a new directory for each restart run and copy the restart files into that directory.
Each run requires the \verb+suntans.dat+ file as well as other files for the grid as well, but rather than
copy these over to the new directory, it is best to create links to the already existing files in order
to save disk space.  In order for the restart run to proceed, links must be created for each of these
files (or copies):
\begin{itemize}
\item suntans.dat
\item celldata.dat.np
\item edgedata.dat.np
\item topology.dat.np
\item vertspace.dat
\end{itemize}
where \verb+np+ corresponds to each processor id.  Considering the previous example, to restart a run
one might create a new directory called \verb+run2+.  Then links could be made to the existing files 
with
\begin{verbatim}
ln -s suntans.dat run2/suntans.dat
ln -s celldata.dat.0 run2/celldata.dat.0
ln -s celldata.dat.1 run2/celldata.dat.1
ln -s edgedata.dat.0 run2/edgedata.dat.0
ln -s edgedata.dat.1 run2/edgedata.dat.1
ln -s topology.dat.0 run2/topology.dat.0
ln -s topology.dat.1 run2/topology.dat.1
ln -s vertspace.dat run2/vertspace.dat
\end{verbatim}
and then restarting SUNTANS (after copying the store files into run2) with
\begin{verbatim}
mpirun -np 2 sun -s -r --datadir=./run2
\end{verbatim}
where the \verb+r+ flag indicates a restart run.  An alternative method to perform a restart
is to create a new directory and then alter the entries in \verb+suntans.dat+ so that the
output files are specified by their full rather than relative paths.  For example, to set
the output directory so that SUNTANS outputs the free surface to the new directory \verb+run2+,
the entry for \verb+FreeSurfaceFile+ in \verb+suntans.dat+ could be changed from
\begin{verbatim}
FreeSurfaceFile   	fs.dat
\end{verbatim}
to
\begin{verbatim}
FreeSurfaceFile   	run2/fs.dat
\end{verbatim}
