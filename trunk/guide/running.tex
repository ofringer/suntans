\section{Running {SUNTANS}}

Before running SUNTANS, you must have a valid triangulation and a valid parameter file \verb+suntans.dat+.
For details on the triangulation, see Section \ref{sec:grids}, and for details on the \verb+suntans.dat+
parameter file, see Section \ref{sec:params}.  If you have a valid grid and parameter file in the same
directory as the \verb+sun+ executable, SUNTANS can be executed as a single processor job 
by running the main executable \verb+sun+ with
\begin{verbatim}
mpirun sun -t -g -s 
\end{verbatim}
Running with the input and output data files in the same directory as the source and executable
is in general not a good idea because SUNTANS creates many
input and output files that will clutter up the local directory.  It is best to create a new
directory and specify that directory as the working data directory on the command line and place the
grid and parameter files in that directory.  For
example, if the input files are in the local \verb+data+ directory, then SUNTANS would be
run with
\begin{verbatim}
mpirun sun -t -g -s --datadir=./data
\end{verbatim}
To run SUNTANS on multiple processors, use the \verb+-np+ flag with \verb+mpirun+. For example,
to run SUNTANS on 64 processors, use
\begin{verbatim}
mpirun -np 64 sun -t -g -s --datadir=./data
\end{verbatim}
The following is a list of flags that determine the behavior of the \verb+sun+ executable upon
running:
\begin{itemize}
\item{\verb+-t+} Create a triangulation from a planar straight line graph.  For details see
Section \ref{sec:grids}.
\item{\verb+-g+} Partition the triangulation among the given number of processors and
compute grid geometry and cell and processor connectivity.  For details see Section \ref{sec:grids}.
\item{\verb+-s+} Run the SUNTANS solver.
\item{\verb+-v[vvv]+} Output information about the progress of the run.  The more \verb+v+s, the
more verbose the output (maximum 4).
\item{\verb+-w+} Print out warnings that may lead to crashes or erroneous results to the screen
(independent of \verb+-v+).
\item{\verb+-r+} Restart SUNTANS from a previous run.  For details see Section \ref{sec:restart}.
\end{itemize}
A typical SUNTANS run with, for example, 4 processors, will proceed as follows:
\begin{enumerate}
\item Place the parameter file \verb+suntans.dat+ and planar straight line graph file in a directory,
say the \verb+data+ directory.
\item Create the triangulation and grid information with
\begin{verbatim}
mpirun -np 4 sun -t -g --datadir=./data
\end{verbatim}
This stores the grid information in files (see Section \ref{sec:grids}) for later reading later.
Since this process may take some time it is a good idea to run this once for large grids and
read in the grid data from a file.  
\item Read the grid data from the files and run the solver and output information about the
run with
\begin{verbatim}
mpirun -np 4 sun -s -vv --datadir=./data
\end{verbatim}
\item Once this run is finished it is possible to restart the run (for details see Section \ref{sec:restart})
with the data in \verb+data+ using
\begin{verbatim}
mpirun -np 4 sun -s -r -vv --datadir=./data
\end{verbatim}
\end{enumerate}

Note that the reason behind being able to specify the data directory at the command line is that
it enables the executable to be run from the same directory but to use data from different directories
that may contain different parameters.  For example, if the \verb+data1+ directory and \verb+data2+ 
directory contain different grid data (from previous calls to SUNTANS with the \verb+-t+ and \verb+-g+ flags),
then SUNTANS can be run either with
\begin{verbatim}
mpirun -np 4 sun -s --datadir=./data1
\end{verbatim}
or with
\begin{verbatim}
mpirun -np 4 sun -s --datadir=./data2
\end{verbatim}
