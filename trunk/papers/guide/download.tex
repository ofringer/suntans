\section{Downloading and installing SUNTANS}

In order to use SUNTANS, you must also install the message-passing
interface (MPI), the parallel graph partitioning libraries (ParMetis~\cite{PARMETIS[1998]}),
the grid generation package Triangle~\cite{TRIANGLE[1996]}.  Instructions for 
downloading and installing these packages are available from the individual websites
for each package:
\begin{tabbing}
MPI\hspace{0.5in}\=  \verb+http://www-unix.mcs.anl.gov/mpi/mpich/+\\
ParMetis \> \verb+http://www-users.cs.umn.edu/~karypis/metis/parmetis/+\\
Triangle \> \verb+http://www-2.cs.cmu.edu/~quake/triangle.html+
\end{tabbing}
Note that you must compile the triangle libraries as object files by making them
with \verb+make trilibrary+.  Currently SUNTANS compiles and runs with Parmetis-2.0.

\subsection{Downloading and installing the latest source} \label{sec:download}

To obtain a copy of SUNTANS, download the latest version from the SUNTANS
web page at \verb+http://suntans.stanford.edu/downloads+.  Follow the following
steps to install the software.  {\bf Note: Do not continue beyond this point unless
you have installed the required packages mentioned in the previous paragraph!
SUNTANS will not compile unless you do so, even if you do not intend to run it in
parallel.}
\begin{enumerate}
\item First unpack the gzipped tar archive with
\begin{verbatim}
tar xzvf suntans-X.X.tgz
\end{verbatim}
where \verb+X.X+ is the latest version of SUNTANS.  
\item In the suntans-X.X directory, edit Makefile.in so that the directories containing the appropriate packages
are correctly specified as follows:
\begin{itemize}
\item \verb+MPIHOME+ should contain the base directory of the mpich distribution.
For example, 
\verb+MPIHOME=/usr/local/mpich-1.2.7+
\item \verb+PARMETISHOME+ should contain the base directory of the ParMetis distribution.
\item \verb+TRIANGLEHOME+ should contain the base directory of the Triangle libraries.
\end{itemize}
Note that there cannot be any spaces between the ``='' sign and the value.  As an example,
the \verb+Makefile.in+ file might look like
\begin{verbatim}
MPIHOME=/usr/local/mpich-1.2.7
PARMETISHOME=/usr/local/packages/ParMetis-2.0
TRIANGLEHOME=/usr/local/packages/triangle
\end{verbatim}
\item Once these locations are properly specified, compile the SUNTANS executable with
\begin{verbatim}
make
\end{verbatim}
This will create the main executable \verb+sun+.
\item Compile the SUNTANS graphical user interface with
\begin{verbatim}
make sunplot
\end{verbatim}
This will create the gui executable \verb+sunplot+.  Note:  This GUI requires the
existence of the \verb+Xlib+ libraries and it is assumed that these are located
in \verb+/usr/X11R6+.  Make sure the \verb+XINC+ and \verb+XLIBDIR+ variables
are specified correctly in the \verb+Makefile+ if \verb+sunplot+ does not compile.
\item In order to remove the object files, use
\begin{verbatim}
make clean
\end{verbatim}
\item To clean up the directory even further (remove \verb+*~, #*#, PI*, sun, sunplot+),
use
\begin{verbatim}
make clobber
\end{verbatim}
\end{enumerate}

\subsection{Keeping up to date with CVS}

You can keep an up to date copy of the SUNTANS distribution on your machine by using the
cvs repository.  To do so, you need to send your ssh public key to the suntans server administrator so that
it can be installed in the authorized keys file on the suntans server.  To obtain your
public key, use
\begin{verbatim}
ssh-keygen -t rsa
\end{verbatim}
This will create a public key in the file \verb+~/.ssh/id_rsa.pub+, which you should send
via email to the administrator.  You may use an empty passphrase but it is not recommended.
Once you send your public key to the suntans cvs administrator, in the \verb+bash+ shell, type
\begin{verbatim}
export CVSROOT=:ext:cvsuser@suntans:/home/cvs
export CVS_RSH=ssh
\end{verbatim}
This sets the \verb+CVSROOT+ environmental variable so that when you use the cvs commands
they look for the cvs repository on the suntans server as user cvsuser.
You will be granted read-only access to the cvs repository, so you can keep an up to date
copy of the latest source on your machine.  To check out the latest copy, type
\begin{verbatim}
cvs checkout suntans
\end{verbatim}
This will create the \verb+suntans+ directory and all of the subdirectories on the server.
If you would like to sync your copy of SUNTANS with the copy on the server, use
\begin{verbatim}
cvs update suntans
\end{verbatim}
Note that you are only allowed read access to the cvs server.  Any changes you make to
the SUNTANS files on your machine will not be updated on the server unless you are added
to the group list on the server.  You will also not be able to ssh into the server if you
try, as this will freeze your screen as only cvs access is allowed via ssh.
