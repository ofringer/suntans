\section{Using the sunplot GUI}

The sunplot GUI is meant to be a debugging tool and hence it does not have the ability to output
results into a figure file, such as postscript, jpeg, or eps.  It is, however, a handy tool to
quickly view results without having to execute third-party software.  Sunplot has the ability to
view all of the data output from SUNTANS on several processors, and it only requires the X11 libraries.

This howto describes how to view the results presented in the two-dimensional river plume
example and assumes that the data has been created in the respective data directory
in \verb+examples/boundaries/data+.  Details on how to create this data is given in 
Section \ref{sec:boundary_ex},
and details on how to compile \verb+sunplot+ are described in Section \ref{sec:download}.

\subsection{Starting up the GUI}

The \verb+sunplot+ GUI can be started while a simulation is running as long as the data files
are not empty.  To view the data with the default settings in a given directory, use
\begin{verbatim}
./sunplot --datadir=examples/boundaries/data
\end{verbatim}
This will bring up a planview of the salinity data at the first time step $n=1$ in the upper layer $k=1$ (note
that the GUI does not use C-style indexing, so the first index starts at k=1).  If more than one processor
is being viewed, that should be specified at the command line, as in
\begin{verbatim}
./sunplot -np 8 --datadir=examples/boundaries/data
\end{verbatim}
For multiprocessor output, the processors being displayed will start at 0 and end at \verb+np-1+. 
For example, for a 64 processor job, the command
\begin{verbatim}
./sunplot -np 3 --datadir=examples/boundaries/data
\end{verbatim}
will show data on processors 0, 1, and 2.

\subsection{Moving around in time}

The three buttons beneath \verb+''Step: 1 of 41''+ on the upper left of the GUI control the time stepping.
\begin{list}{}
\item \button{$<--$}
\begin{list}{}
\item {\bf Left button:} Moves one step backward in time.
\item {\bf Middle button:} No effect
\item {\bf Right button:} Moves to time step 1.
\end{list}
\item \button{M}: 
\begin{list}{}
\item {\bf Left button:} Animates forward in time from begining to end.
\item {\bf Middle button:} Asks the user to input a desired time step at the command line.
\item {\bf Right button:} Animates backward in time from the end to the begining.
\end{list}
\item \button{$-->$};
\begin{list}{}
\item {\bf Left button:} Moves one step forward in time.
\item {\bf Middle button:} No effect
\item {\bf Right button:} Moves to the last step (in this case step 41).
\end{list}
\end{list}

\subsection{Displaying different vertical levels}

The two buttons beneath \verb+''Level: 1 of 20''+ on the upper left of the GUI control the vertical level
being shown.  Level 1 is the top level, while level 20 is level $N_{kmax}$.
\begin{list}{}
\item \button{$<--$}
\begin{list}{}
\item {\bf Left button:} Displays level up (decreasing k).
\item {\bf Middle button:} No effect
\item {\bf Right button:} Displays level 1.
\end{list}
\item \button{$-->$}:
\begin{list}{}
\item {\bf Left button:} Displays one level down (increasing k).
\item {\bf Middle button:} No effect
\item {\bf Right button:} Displays level $N_{kmax}$.
\end{list}
\end{list}

\subsection{Displaying different processors}

The three buttons beneath \verb+''Showing all 1 Procs''+ on the upper left of the GUI control the processor
being shown.  The default is to show all processors.  
\begin{list}{}
\item \button{$<--$}
\begin{list}{}
\item {\bf Left button:} Decreases the processor being displayed.  Loops around if on processor 1.
\item {\bf Middle button:} No effect
\item {\bf Right button:} Displays processor 1 (no C-style indexing).
\end{list}
\item \button{All}:
\begin{list}{}
\item {\bf Left button:} Displays all processors.
\item {\bf Middle button:} No effect.
\item {\bf Right button:} No effect.
\end{list}
\item \button{$-->$}:
\begin{list}{}
\item {\bf Left button:} Increases the processor number being displayed.  Loops around if on the last processor.
\item {\bf Middle button:} No effect
\item {\bf Right button:} Displays the last processor.
\end{list}
\end{list}

\subsection{Surface plots of data}

The default quantity to display is the salinity, or s, in planview, at level 1 and time step 1.  Color axes
are scaled to the minimum and maximum of the profile being shown.  They may also be specified by
parameters in \verb+suntans.dat+, as described in Section \ref{sec:caxis}. The following
buttons control their respective plots:
\begin{list}{}
\item \button{S}: Salinity
\begin{list}{}
\item {\bf Left button:} Displays the salinity $s$.
\item {\bf Middle button:} Displays the salinity anomoly $s-s_0$.
\item {\bf Right button:} Displays the background salinity $s_0$.
\end{list}
\item \button{T}: Temperature
\begin{list}{}
\item {\bf Left button:} Displays the temperature $T$.
\item {\bf Middle button:} No effect.
\item {\bf Right button:} No effect.
\end{list}
\item \button{q}: Nonhydrostatic pressure
\begin{list}{}
\item {\bf Left button:} Displays the nonhydrostatic pressure $q$.
\item {\bf Middle button:} No effect.
\item {\bf Right button:} No effect.
\end{list}
\item \button{h}: Free surface
\begin{list}{}
\item {\bf Left button:} Displays the free surface $h$.
\item {\bf Middle button:} No effect.
\item {\bf Right button:} No effect.
\end{list}
\item \button{n}:  Eddy viscosity 
\begin{list}{}
\item {\bf Left button:} Displays the eddy viscosity $\nu_t$ at the center of the cell.
\item {\bf Middle button:} No effect.
\item {\bf Right button:} No effect.
\end{list}
\item \button{k}: Scalar diffusivity
\begin{list}{}
\item {\bf Left button:} Displays the scalar diffusivity $\kappa_t$ at the center of the cell.
\item {\bf Middle button:} No effect.
\item {\bf Right button:} No effect.
\end{list}
\item \button{U}: 
\begin{list}{}
\item {\bf Left button:} Displays the horizontal cartesian velocity component $u$, at the center of the cell.
\item {\bf Middle button:} No effect.
\item {\bf Right button:} Displays the horizontal cartesian baroclinic velocity component $u_b$, at the center of the cell.
This is calculated with
\[
u_b = u - \frac{1}{D}\int_{-D}^h u\,dz\,.
\]
\end{list}
\item \button{V}: 
\begin{list}{}
\item {\bf Left button:} Displays the horizontal cartesian velocity component $v$, at the center of the cell.
\item {\bf Middle button:} No effect.
\item {\bf Right button:} Displays the horizontal cartesian baroclinic velocity component $v_b$, at the center of the cell.
This is calculated with
\[
v_b = v - \frac{1}{D}\int_{-D}^h v\,dz\,.
\]
\end{list}
\item \button{W}: 
\begin{list}{}
\item {\bf Left button:} Displays the vertical cartesian velocity component $w$, at the center of the cell.
\item {\bf Middle button:} No effect.
\item {\bf Right button:} No effect.
\end{list}
\item \button{Depth}: 
\begin{list}{}
\item {\bf Left button:} Displays the depth d beneath the 0 datum.
\item {\bf Middle button:} No effect.
\item {\bf Right button:} Displays the water depth $h+d$.
\end{list}
\item \button{None}: 
\begin{list}{}
\item {\bf Left button:} No surface plot is shown.  Useful for only showing vector field.
\item {\bf Middle button:} No effect.
\item {\bf Right button:} No effect.
\end{list}
\end{list}

\subsection{Vector plots}

Vector plotting is controlled with the Vectors, iskip, and kskip buttons.
\begin{list}{}
\item \button{Vectors}:
\begin{list}{}
\item {\bf Left button:} Toggle vectors on/off.
\item {\bf Middle button:} Shorten the vectors by half their length.
\item {\bf Right button:} Double the length of the vectors.
\end{list}
\begin{list}{}
\item \verb+iskip/kskip+: Number of indices to skip to clarify the vector plot.
When in planview mode of the data, only every \verb+iskip+ vector on the unstructured
grid will be shown. Changing \verb+kskip+ has no
effect on the planview plot.
In profile mode, every \verb+iskip+ vector in the x-direction is shown and 
every \verb+kskip+ vector in the z-direction is shown.  The value of \verb+iskip+
and \verb+kskip+ are changed with their respective arrow buttons:
\begin{list}{}
\item \button{$<$}: Decrease skip amount.
\begin{list}{}
\item {\bf Left button:} Decrease skip by 1.
\item {\bf Middle button:} No effect.
\item {\bf Right button:} Set skip amount to 1.
\end{list}
\item \button{$>$}: Increase skip amount.
\begin{list}{}
\item {\bf Left button:} Increase skip by 1.
\item {\bf Middle button:} No effect.
\item {\bf Right button:} Planview mode: set skip amount to 5.  In profile mode, set skip
amount to maximum index size.
\end{list}
\end{list}
\end{list}
\end{list}

\subsection{Displaying the grid}

The grid can be viewed with the following buttons:
\begin{list}{}
\item \button{Edges}:
\begin{list}{}
\item {\bf Left button:} Toggle display of Delaunay edges.
\item {\bf Middle button:} No effect.
\item {\bf Right button:} No effect.
\end{list}
\item \button{Voro}:
\begin{list}{}
\item {\bf Left button:} Toggle display of Voronoi points.
\item {\bf Middle button:} No effect.
\item {\bf Right button:} No effect.
\end{list}
\item \button{Dela}:
\begin{list}{}
\item {\bf Left button:} Toggle display of Delaunay points.
\item {\bf Middle button:} No effect.
\item {\bf Right button:} No effect.
\end{list}
\end{list}

\subsection{Zooming in on data and obtaining profile plots}

By default, zooming is on.  The effect of using the mouse and selecting a point on the
plotted data depends on whether zooming is on or off or if you are viewing a profile plot.
\begin{list}{}
\item \button{ZOOM}
\begin{list}{}
\item {\bf Left button:} Toggle turning zoom on/off.  
\begin{list}{}
\item When zooming is {\bf on}:
\begin{list}{}
\item {\bf Left button:}  Can be used either to select an area in the plot to zoom into, or
it can be used to zoom into an area by a factor of 2, centered about the clicking point.
\item {\bf Middle button:} Returns to the default in which the entire domain is displayed 
provided that there is no mouse movement.  If there is mouse movement, the view will pan 
corresponding to the mouse motion on the button release.  Alternatively, the user can toggle 
a panning mode via pressing the `m' key.
\item {\bf Right button:} Zooms out by a factor of 2.
\end{list}
\item When zooming is {\bf off}:
\begin{list}{}
\item {\bf Left button:}  Used to select a line along which a 2-d x-z transect of the data
is displayed.  Note that the transect uses a simple nearest-neighbor interpolation to
display the data in the transect.  If you press the mouse button once in the domain
while zooming is off, you may get a message that says \verb+''Cannot plot this slice''+.
This occurs because the length of the transect you have chosen is effectively zero by
clicking once on the plot window when zooming is off.
\item {\bf Middle button:} No effect.
\item {\bf Right button:} Displays the processor number and the value of the data at the selected point.
\end{list}
\end{list}
\item {\bf Middle button:} No effect.
\item {\bf Right button:} No effect.
\end{list}
\item \button{Profile}
\begin{list}{}
\item {\bf Left button:} This will toggle back and forth between the surface plot and the previous
transect that was displayed.  In the event that no transect has been displayed, the error
message \verb+Need two points for a profile first!+ will be displayed.  In this case, turn
zooming off and select two points for a profile.
\item {\bf Middle button:} Display the default west-east transect along the middle of the data set.
This is useful when you have a channel flow in the x-direction and want to see the data in
the x-z plane.
\item {\bf Right button:} Display the default north-south transect along the middle of the data set.
This is useful when you have a channel flow in the x-direction and want to see the cross-channel data in
the x-y plane.
\end{list}
\end{list}

\subsection{Changing the axes scaling}

By default, the x-y axes are scaled equally in ``Image'' mode (If the
\verb+-2d+ flag is supplied at the command line, however, the default mode is ``Normal'').  This
makes profile plots in x-z difficult to view especially for high
aspect ratio grids.  Hitting the \button{Aspect} button with the left
mouse button will toggle between ``image'' mode and ``normal'' mode,
in which the data is scaled to fit the plot area.  You can also do
this by hitting the ``a'' key.  The middle and right mouse buttons have
no effect here.

\subsection{Changing the color axes} \label{sec:caxis}

By default, the color axes are scaled with the maximum and minimum values of the data.
This can be changed with the Caxis button as follows:
\begin{list}{}
\item \button{Caxis}
\begin{list}{}
\item {\bf Left button:} Toggle between scaling with the maximum and minimum data values and
using the values specified in \verb+suntans.dat+.  Specifically, if you add the following
lines to your \verb+suntans.dat+ file, the data can be scaled with these color axes values
when pressing the Caxis button:
\begin{verbatim}
caxismin   -0.25      # Minimum value for color axes
caxismax    0.25      # Maximum value for color axes
\end{verbatim}
If these values are not in \verb+suntans.dat+, an error will be indicated each time data
is plotted until the Caxis button is hit again to return to the default scaling mode.
\item {\bf Middle button:} No effect.
\item {\bf Right button:}  Enter color axes values at the command line.  The format must be
in the form \verb+cmin cmax+ followed by a carriage return.  Carriage returns may also
exist between the entries of \verb+cmin+ and \verb+cmax+.
\end{list}
\end{list}

\subsection{Quitting out of sunplot}

You can either hit the \button{Quit} button or the ``q'' key to quit out of sunplot.

\subsection{Command line options}

Options can be passed to sunplot at the command line to reduce the number of key strokes
required to obtain a desired view of data.  The following is a list of command line arguments:
\begin{list}{}
\item \verb+--datadir=path+: This is the path to the directory containing the sunplot.dat
data file.  If this is not specified, then it is assumed that the data file is in the local
directory.
\item \verb+-np 2+: Specify the number of processors included in the data.  For a multi-processor
data set, if this is not specified then only the data from processor \verb+0+ will be displayed.
\item \verb+-2d+: Display an x-z profile plot of the data in axis Normal mode.  This is equivalent
to starting sunplot with no command line options and hitting the Axes button with the left mouse
button and the Profile button with the middle mouse button.
\item \verb+-g+: Only display the grid.  This is useful for debugging the grid without loading
in any other data.
\item \verb+-m+: Step forward through the data as an animation.
\end{list}
