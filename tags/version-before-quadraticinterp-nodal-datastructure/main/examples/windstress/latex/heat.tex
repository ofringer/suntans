\documentclass[12pt]{article}

\usepackage{amsfonts,graphics,amsmath}

\textwidth 6.5in
\textheight 9.0in
\topmargin 0.0in
\oddsidemargin 0.0in
\evensidemargin 0.0in

%
% New command file
%

% Partial Derivative
\newcommand{\D}[2]{\frac{\partial{#1}}{\partial{#2}}}
\newcommand{\dd}[2]{\frac{d{#1}}{d{#2}}}
\newcommand{\ND}[3]{\frac{\partial^{#3}{#1}}{\partial{#2}^{#3}}}
\newcommand{\DHO}[2]{\frac{\partial^H{#1}}{\partial{#2}}}
\newcommand{\de}[2]{\frac{\delta{#1}}{\delta{#2}}}
\newcommand{\deln}[3]{\frac{\delta^#3 #1}{\delta #2^#3}}
\newcommand{\SD}[2]{\frac{\partial^2{#1}}{\partial{#2}\partial{#2}}}
\newcommand{\order}[1]{\mathcal{O}\left(#1\right)}
\newcommand{\etal}{{\it\space et al. }}
\newcommand{\ub}{\mathbf{u}}
\newcommand{\fb}{\mathbf{f}}
\newcommand{\xb}{\mathbf{x}}
\newcommand{\ut}{\utilde{u}}
\newcommand{\ol}[1]{\overline{#1}}
\newcommand{\twocol}[2]{\parbox{2in}{#1}\parbox{2in}{#2}}
%
% \insertfig{scalefactor}{epsfile}{caption}{label}
%
\newcommand{\insertfig}[4]{
	\begin{figure}[ht!]
	\centerline{
	\scalebox{#1}{\includegraphics{#2}}
	}
	\caption{#3}
	\label{#4}
	\end{figure}}


% simd

\begin{document}

\title{Nonpenetrative heat flux implementation in SUNTANS}

\author{Oliver Fringer}

\maketitle

We will assume that the nonpenetrative heat fluxes are imposed at the surface
as flux boundary conditions and not as source terms. As described below,
this is equivalent to imposing a source term at the top layer and assuming
adiabatic boundaries.  The implementation 
is based on the description in Wood et al., 2008, 
"Modeling hydrodynamics and heat transport in upper Klamath Lake, Oregon, 
and implications for water quality", USGS Scientific Investigations 
Report 2008-5076.

\section{General form of the linearized heat flux boundary condition}

Given a nonpenetrative flux $H_{NP}$, the top boundary condition is given by
\[
\kappa_T\left.\D{T}{z}\right|_{Top} = -\frac{1}{\rho c_p}H_{NP}\,.
\]
We can derive a linearized form for a heat flux that is a nonlinear function
of the surface water temperature by
linearizing the heat flux about the temperature at time step $n$, $T^n$
such that $T_w = T^n + T'$. This gives, to $O((T')^2)$,
\begin{eqnarray*}
H_{NP}(T^n  + T') &=& H_{NP}|_{T_w=T^n} + \left.\D{H_{NP}}{T}\right|_{T_w=T^n}T'\,,\\
                       &=& H_{NP}|_{T_w=T^n} + \left.\D{H_{NP}}{T}\right|_{T=T^n}\left(T_w - T^n\right)\,,\\
                       &=& H_{NP}|_{T_w=T^n} - \left.\D{H_{NP}}{T}\right|_{T=T^n}T^n
                                             + \left.\D{H_{NP}}{T}\right|_{T=T^n}T_w\,,\\
                       &=& H_{NP,0} + \Delta H_{NP,0} T_w\,,
\end{eqnarray*}
where 
\begin{eqnarray*}
H_{NP,0} &=& H_{NP}|_{T_w=T^n} - \left.\D{H_{NP}}{T}\right|_{T=T^n}T^n\,,\\
\Delta H_{NP,0} &=& \left.\D{H_{NP}}{T}\right|_{T=T^n}\,.
\end{eqnarray*}
The linearized top boundary condition is then given by
\[
\kappa_T\left.\D{T}{z}\right|_{top} = -\frac{1}{\rho_0 c_p}H_{NP} = -\frac{1}{\rho_0 c_p}H_{NP,0} - \frac{1}{\rho_0 c_p}\Delta H_{NP,0} T = -a - b T\,,
\]
where 
\begin{eqnarray*}
a &=& \frac{1}{\rho_0 c_p}H_{NP,0}\,,\\
b &=& \frac{1}{\rho_0 c_p}\Delta H_{NP,0}\,.
\end{eqnarray*}
The heat equation in the absence of advection or source terms is given by
\[
\D{T}{t} = \D{}{z}\left(\kappa_T\D{T}{z}\right)\,,
\]
which is discretized in time with the theta method
\[
\frac{T^{n+1} - T^n}{\Delta t} = \theta\D{}{z}\left(\kappa_T\D{T^{n+1}}{z}\right)
                              +(1-\theta)\D{}{z}\left(\kappa_T\D{T^{n}}{z}\right)\,.
\]
At the top boundary, we have
\begin{eqnarray*}
\D{}{z}\left(\kappa_T\D{T}{z}\right)_{top} &=&
\frac{1}{\Delta z_{top}}\left(\kappa_T\left.\D{T}{z}\right|_{top} 
- \kappa_T\left.\D{T}{z}\right|_{top+1}\right)\,,\\
                                     &=&
-\frac{1}{\rho_0 c_p \Delta z_{top}}H_{NP}
+\D{}{z}\left(\kappa_T\D{T^{n+1}}{z}\right)_{top,adiabatic}\,,
\end{eqnarray*}
where the adiabatic subscript implies use of the diffusion operator at the
top cell with the assumption of an adiabatic top boundary.  This shows
that application of a flux condition at the top boundary is equivalent to
assuming an adiabatic top boundary and adding a source term of the form
\[
F_S = -\frac{1}{\rho_0 c_p \Delta z_{top}}H_{NP} = -\frac{a}{\Delta z_{top}} - \frac{b}{\Delta z_{top}} T_{top}
\]
to the top cell.
Therefore, the governing equation for temperature at the top cell is given by
\begin{eqnarray*}
\D{T_{top}}{t} &=& -\frac{1}{\Delta z_{top}}\left(a + b T_{top}\right)
               +\D{}{z}\left(\kappa_T\D{T}{z}\right)_{top,adiabatic}\,,\\
               &=& A + B T_{top}
               +\D{}{z}\left(\kappa_T\D{T}{z}\right)_{top,adiabatic}\,,
\end{eqnarray*}
where
\begin{eqnarray*}
A &=& -\frac{1}{\rho_0 c_p \Delta z_{top}}H_{NP,0}\,,\\
B &=& -\frac{1}{\rho_0 c_p \Delta z_{top}}\Delta H_{NP,0}\,.
\end{eqnarray*}
The derivatives with respect to temperature for the individual heat fluxes
are derived below. 

\section{Longwave emitted radiation}

The longwave emitted radiation is given by
\[
H_{LE} = \alpha_{LE}\left(T_w + T_0\right)^4\,,
\]
where $\alpha_{LE}=\epsilon_w\sigma$.
Differentiating with respect to temperature gives
\begin{eqnarray*}
\left.\D{H_{LE}}{T}\right|_{T=T_0 + T^n} &=& 4\alpha_{LE}\left(T_0 + T^n\right)^3 \,,\\
                                         &=& \frac{4H_{LE}}{T_0 + T^n}\,.
\end{eqnarray*}

\section{Evaporation}

Evaporative heat loss is given by
\[
H_E = \alpha_E (e_s - e_a)\,,
\]
where $\alpha_E = \rho_0 L_w F$, and where $L_w$ is the latent heat of evaporation, 
\[
F = \alpha_{E1} (\alpha_{E2} + \alpha_{E3} U_2)
\]
accounts for the effects of the magnitude of the wind, $U_2$, a distance
of 2~m above the surface, 
\[
e_s = \alpha_{E4} \exp\left(\frac{\alpha_{E5} T_w}{T_w + T_{00}}\right)
\]
is the saturation vapor pressure at the water surface, and
\[
e_a = \alpha_{E4} \exp\left(\frac{\alpha_{E5} T_a}{T_a + T_{00}}\right) rh
\]
is the vapor pressure in the air that has relative humidity rh (as a fraction of 1 not \%!),
and $T_{00}=237.3$~K.
The constants of proportionality are denoted by $\alpha_{En}$, $n=1,2,\dots,6$.  Differentation with
respect to $T$ gives
\begin{eqnarray*}
\D{H_E}{T} &=& \alpha_E\D{e_s}{T}\,,
\end{eqnarray*}
where 
\begin{eqnarray*}
\D{e_s}{T} &=& \alpha_{E4}\exp\left(\frac{\alpha_{E5} T_w}{T_w + T_{00}}\right)
\left[\frac{\alpha_{E5}T_{00}}{\left(T_w + T_{00}\right)^2}\right]\,,\\
           &=& \alpha_{E5}e_s\frac{T_{00}}{\left(T_w + T_{00}\right)^2}\,.
\end{eqnarray*}
This gives
\[
\left.\D{H_E}{T}\right|_{T=T_w} = \alpha_E\alpha_{E5}e_s\frac{T_{00}}{\left(T_w + T_{00}\right)^2}\,.
\]

\section{Sensible heat flux}

The sensible heat flux is given by
\[
H_S = H_E C_B \frac{p_a}{p_{sl}}\left(\frac{T_w - T_a}{e_s - e_a}\right)\,,
\]
where $C_B$ is an empirical coefficient, $p_a$ is the atmospheric pressure, and
$p_sl$ is the sea-level pressure. The derivative is given by
\begin{eqnarray*}
\D{H_S}{T} &=& \D{H_E}{T} C_B \frac{p_a}{p_{sl}}\left(\frac{T_w - T_a}{e_s - e_a}\right)
+ H_E C_B \frac{p_a}{p_{sl}}
\frac{e_s - e_a - \alpha_{E5}e_s(T_w-T_a)\frac{T_{00}}{\left(T_w + T_{00}\right)^2}}{(e_s-e_a)^2}\,,\\
&=& \D{H_E}{T} C_B \frac{p_a}{p_{sl}}\left(\frac{T_w - T_a}{e_s - e_a}\right)
+ H_E C_B \frac{p_a}{p_{sl}}\left(\frac{T_w-T_a}{e_s-e_a}\right)
\frac{e_s - e_a - \alpha_{E5}e_s(T_w-T_a)\frac{T_{00}}{\left(T_w + T_{00}\right)^2}}{(T_w-T_a)(e_s-e_a)}\,,\\
&=& \frac{1}{H_E}\D{H_E}{T} H_E C_B \frac{p_a}{p_{sl}}\left(\frac{T_w - T_a}{e_s - e_a}\right)
+ H_E C_B \frac{p_a}{p_{sl}}\left(\frac{T_w-T_a}{e_s-e_a}\right)
\frac{e_s - e_a - \alpha_{E5}e_s(T_w-T_a)\frac{T_{00}}{\left(T_w + T_{00}\right)^2}}{(T_w-T_a)(e_s-e_a)}\,,\\
&=& H_S\left[\frac{1}{H_E}\D{H_E}{T}+
\frac{e_s - e_a - \alpha_{E5}e_s(T_w-T_a)\frac{T_{00}}{\left(T_w + T_{00}\right)^2}}{(T_w-T_a)(e_s-e_a)}\right]\,,\\
&=& H_S\left[\frac{1}{H_E}\D{H_E}{T}+
\frac{1}{T_w-T_a} - \alpha_{E5}\frac{e_s}{(e_s-e_a)}\frac{T_{00}}{\left(T_w + T_{00}\right)^2}\right]\,,\\
\end{eqnarray*}





\end{document}