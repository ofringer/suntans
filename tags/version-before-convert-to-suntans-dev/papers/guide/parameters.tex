\section{Parameter listing in suntans.dat} \label{sec:params}

The main SUNTANS input file is called \verb+suntans.dat+  and contains three
columns.  The first column contains the name of the variable used in
the program while the second column contains its value, and the information
after that value is ignored.  Lines beginning
with \verb #  are ignored, while {\bf empty lines will cause the program to 
crash}.  The SUNTANS executable looks for this file in the directory specified
by \verb+--datadir=+.  If this is not specified, then it looks for it in the
same directory as the \verb+sun+ executable.

\subsection{Physical/Computational parameters}

These parameters define values used in SUNTANS.  The syntax of the line in \verb+suntans.dat+ is given by
\begin{verbatim}
variablename  variablevalue   # Description of variable
\end{verbatim}
When this file is read, the effect is equivalent to \verb+variablename=variablevalue+.
This file cannot have empty lines, but the spacing between variablename and value is arbitrary, as long
as it does not contain a new line.  
Only the first variable is read before the \verb+#+ sign.  That is, old variable values can exist to
the right of the variable values being used and they are neglected.  The following line is therefore legal:
\begin{verbatim}
dt    0.025  0.05 (unstable)  0.0125 (10/12/04)  # Time step size
\end{verbatim}
This will have the effect of \verb+dt=0.025+ and neglecting the rest of the line after \verb+0.025+.

\subsubsection{Nkmax: $N_{kmax}\ge 1$}

This is the number of cells in the vertical and is used when the -g flag is
specified in order to build the vertical grid levels.  It must be
greater than or equal to 1.  

\subsubsection{stairstep: Boolean}

If stairstep is 1 then the vertical grid spacing does not change in
the horizontal. If stairstep is 0, then partial-stepping is employed and
the height of the bottom cell
is set such that the bottom face of the cell coincides with the actual
value of the depth. Otherwise (when stairstep=1), the bottom face of
of the lowermost cell is always greater than the actual depth.  

\subsubsection{rstretch: $-1.1 < r < -1$ or $1 < r < +1.1$}

Employ vertical grid stretching such that the distribution of vertical
grid spacings is given by
\[
\Delta z_k = r\,\Delta z_{k+1}\,,
\]
where $0\le \mbox{k} < \mbox{Nkmax}$.  If $r<0$ then the grid is refined near the
bottom, otherwise it is refined at the free-surface.

\subsubsection{CorrectVoronoi: Boolean}

Typical triangulations will always contain degenerate triangles in which
one of the angles are close to or greater than 90 degrees.  For near-right neighboring
triangles, the distance between their Voronoi points can be quite small (zero when
two right triangles abut one another), while for obtuse triangles,
the Voronoi point lies outside the triangle.  This can be corrected in one of two ways.

In the first method, CorrectVoronoi is 1, and the
Voronoi points are corrected if they are too close to each other
relative to the distance between the centroids of the neighboring triangles.
If triangle $1$ and $2$ have centroids defined by $(xc_1,yc_1)$ and $(xc_2,yc_2)$
and Voronoi points defined by $(xv_1,yv_1)$ and $(xv_2,yv_2)$, then the distance
between the centroids is given by
\[
D_c^2 = (xc_2-xc_1)^2 + (yc_2-yc_1)^2\,,
\]
and the distance between the Voronoi points is given by
\[
D_v^2 = (xv_2-xv_1)^2 + (yv_2-yv_1)^2\,.
\]
The Voronoi points are corrected if $D_g/D_c< V_r$, where $V_r$ is the \verb+VoronoiRatio+ parameter,
and if this is the case, they are updated with
\begin{eqnarray}
xv_1 &=& xc + V_r(xc_1-xc)\,,\nonumber\\
yv_1 &=& yc + V_r(yc_1-yc)\,,\nonumber\\
xv_2 &=& xc + V_r(xc_2-xc)\,,\nonumber\\
yv_2 &=& yc + V_r(yc_2-yc)\,,\nonumber
\end{eqnarray}
where $xc=(xc_1+xc_2)/2$ and $yc=(yc_1+yc_2)/2$.  Using this correction methodology,
setting $V_r=1$ moves the Voronoi points to the centroids of the cells.

In the second method, CorrectVoronoi is set to -1 and triangles
with angles greater than or equal to the angle defined by VoronoiRatio are corrected by moving the Voronoi point
to the triangle centroid.  While the centroid is guaranteed to be inside the triangle and the distance between
centroids is on the same order as the triangle edge lengths, the disadvantage is that the Voronoi edge connecting
the Voronoi points is no longer perpendicular to the Delaunay edges.  This reduces the accuracy of the gradients
defined by differences between values at Voronoi points because it is assumed that the Voronoi-Delaunay intersection
is orthogonal.

\subsubsection{VoronoiRatio}

Adjustment factor for degenerate triangles.  Use of this parameter depends on the value of 
CorrectVoronoi defined above:

\medskip
\noindent
if CorrectVoronoi=1, $0\le \mbox{VoronoiRatio} \le 1$: Voronoi points are moved by this amount towards the
triangle centroids.

\medskip
\noindent
if CorrectVoronoi=-1 $0\le \mbox{VoronoiRatio} \le 90$: Move Voronoi points to centroids for triangles with
angles greater than or equal to VoronoiRatio.

\medskip
\noindent
if CorrectVoronoi=0:no correction is performed.

\subsubsection{vertgridcorrect: Boolean}

For an equispaced vertical grid, the vertical grid spacing is constant and
is given by $\Delta z=D_{max}/N_{kmax}$, where $D_{max}$ is the maximum depth in the domain.
In the rare case for which the minimum depth, $D_{min}$ in the domain is less than $\Delta z$, there
are not enough grid points to resolve the region with the smallest depth.  If this is the
case and vertgridcorrect
is set to 1, then the number of vertical grid cells is adjusted such that $\Delta z=D_{min}$,
such that $N_{kmax}=D_{max}/D_{min}$.

\subsubsection{IntDepth: Boolean}

If set to 1, then the depths at the Voronoi points are interpolated from the
depth specified by the depth file (see below).  Otherwise, the depths are
specified in the file \verb+initialization.c+.

\subsubsection{dzsmall: No longer used}

\subsubsection{scaledepth: Boolean}

For debugging purposes.  If set to 1, scales the depth by scaledepthfactor (see below).

\subsubsection{scaledepthfactor}

If scaledepth=1, this scales the depth by the given amount and hence
can be used to reduce the depth and hence the surface gravity wave speed by
a desired amount.

\subsubsection{thetaramptime: $\tau_\theta >0$}

This is used to damp out transient barotropic motions by initializing
the value of $\theta$ in the free-surface solver to 1 and ramping it
down to the value of $\theta$ specified below.  The $\tau_\theta$ parameter
determines the timescale, in seconds, over which this is damped, such that, if the value
of $\theta$ specified below is given by $\theta_0$, then the value of
$\theta$ that is used in the calculations is given by
\begin{eqnarray}
\theta(t) &=& \theta_0 + (1-\theta_0)\exp\left(-\frac{t}{\tau_\theta}\right)\nonumber\,.
\end{eqnarray}

\subsubsection{beta: $\beta\ge 0$}

Expansivity of salt.  This is used in the gravity term on the right hand side of
the Navier-Stokes equations in the form
\[
-g\beta\left(s-s_0\right)\,,
\]
where $s(x,y,z,t)$ is the salinity and $s_0(z)$ is the initial salinity.  If $\beta=0$
advection of salinity is not computed.  


\subsubsection{theta: $0<\theta\le 1$}

Implicitness parameter for the free-surface solver and the vertial diffusion terms.
\begin{itemize}
\item $\theta=0$: Fully explicit (unstable)
\item $\theta=0.5$: Crank-Nicolson (neutrally stable in the linear sense)
\item $\theta=1.0$: Fully implicit
\item $\theta=0.55$: For practical simulations
\end{itemize}

\subsubsection{thetaS: $0\le \theta_S \le 1$}

Implicitness parameter for vertical scalar advection and diffusion.  Note that this
value of thetaS does not necessarily share the same stability properties as the
free-surface solver since stability for scalar transport is governed by
the horizontal Courant number even when $\theta_S>0.5$.
\begin{itemize}
\item $\theta_S=0$: Fully explicit 
\item $\theta_S=0.5$: Crank-Nicolson (Ensures no vertical Courant limitation for advection or diffusion)
\item $\theta_S=1.0$: Fully implicit (same as above but first-order accurate)
\item $\theta_S=\theta$: To guarantee consistency with continuity 
\end{itemize}

\subsubsection{thetaB: Not used}

\subsubsection{kappa\_s: $\kappa_s\ge 0$}

Laminar vertical diffusivity of salt in units of m$^2$ s$^{-1}$.  Does not affect stability.

\subsubsection{kappa\_sH: $\kappa_{sH}\ge 0$}

Laminar horizontal diffusivity of salt in units of m$^2$ s$^{-1}$.  For stability 
\[
\frac{\kappa_{sH}\Delta t}{\mbox{min}(D_g^2)}<\frac{1}{2}\,,
\]
where $D_g$ is the distance between Voronoi points.  

\subsubsection{gamma: $\gamma$: Boolean}

Temperature is implemented as a passive scalar.  If gamma is set to 0 then scalar
transport of T is not computed.

\subsubsection{kappa\_T: $\kappa_T\ge 0$}

Same as $kappa\_s$ but for T.

\subsubsection{kappa\_TH: $\nu_{TH}\ge 0$}

Same as $kappa\_TH$ but for T.

\subsubsection{nu: $\nu\ge 0$}

Vertical diffusivity of momentum in units of m$^2$ s$^{-1}$.  Does not affect stability unless
central-differencing is employed for momentum advection (see Section \ref{sec:nonlinear}).

\subsubsection{nu\_H: $\nu_{H}\ge 0$}

Horizontal diffusivity of momentum in units of m$^2$ s$^{-1}$.  See \ref{sec:dt} and \ref{sec:nonlinear}
for details on stability.

\subsubsection{tau\_T: $\tau_T$}

Parameter used in boundaries.c to employ a surface wind stress in units of m$^2$ s$^{-2}$,
i.e. $\mbox{tau\_T}=(u^*)^2$.  Note that if the surface wind stress due to a wind speed at 10 m
is given by
\begin{eqnarray*}
\tau_{\mbox{wind}} = \rho_{\mbox{air}}C_d U_{10}^2\,,
\end{eqnarray*}
then to obtain the value for \mbox{tau\_T}, one must use
\begin{eqnarray*}
\mbox{tau\_T} = \frac{\tau_{\mbox{wind}}}{\rho_0}\,.
\end{eqnarray*}

\subsubsection{z0T: $z_{0T}$}

Lid roughness used to calculate drag at the lid (in meters).  See CdT.

\subsubsection{z0B: $z_{0B}$}

Bottom roughness used to calculate drag at the bottom boundary layer (in meters).  See CdB.

\subsubsection{CdT: $C_{dT}$}

If the roughness at the lid is zero (i.e. z0T=0) then the drag coefficient at the lid is given by
this constant, otherwise the drag coefficient is calculated with the rough wall 
relationship
\[
C_{dT} = \left[\frac{1}{\kappa}\ln\left(\frac{1}{2}\frac{\Delta z_{top}}{z_{0T}}\right)\right]^{-2}\,.
\]
where $\Delta z_{top}$ is the thickness of the top cell (in meters) and $\kappa=0.4$ is the von Karman constant.

\subsubsection{CdB: $C_{dB}$}

If the roughness at the bottom is zero (i.e. $z_{0B}$=0) then the drag coefficient at the bottom is given by
this constant, otherwise the drag coefficient is calculated with the rough wall
relationship
\[
C_{dB} = \left[\ln\left(\frac{1}{2}\frac{\Delta z_{bot}}{z_{0B}}\right)\right]^{-2}\,,
\]
where $\Delta z_{top}$ is the thickness of the bottom cell (in meters) and $\kappa=0.4$ is the von Karman constant.

\subsubsection{CdW: $C_{dW}$}

Constant drag coefficient of sidewalls.  Drag on sidewalls is only computed if the wall
does not extend up to the free surface.  For example, in a compound channel, sidewall
drag is only computed within the channel and not on the edges of the floodplain, as shown
in Figure \ref{fig:compoundchannel}.
\insertfig{0.8}{figures/compoundchannel}{Example of a compound channel, for which sidewall
drag is only computed in the channel and not on the shoals.}{fig:compoundchannel}

\subsubsection{turbmodel: Boolean}

If turbmodel is 1 then the Mellor-Yamada level 2.5 turbulence model is used, otherwise
no turbulence model is employed.  

\subsubsection{dt: $\Delta t$} \label{sec:dt}

Time step. Assuming $\theta>0.5$, stability is limited by
\[
\Delta t < C\left[\frac{\min(D_g)}{\max(|U|)},\frac{\min(\Delta z)}{\max(|w|)},
\frac{\min(D_g)}{\max(c_i)},
\frac{1}{2}\frac{\min(D_g)^2}{\max(\kappa_{sH})},
\frac{1}{2}\frac{\min(D_g)^2}{\max(\kappa_{TH})},
\frac{1}{2}\frac{\min(D_g)^2}{\max(\nu_{H})}
\right]\,,
\]
where $c_i$ is the internal wave speed, $C$ is the relevant Courant number, and $D_g$
is the minimum Voronoi distance.  Note that this assumes first-order upwind is used
for advection of momentum.  If central-differencing is used, stability is further
constrained by the grid Peclet number (see Section \ref{sec:nonlinear} for details).

\subsubsection{Cmax: $C_{max}>0$}

Maximum permissible Courant number for advection of momentum.  

\subsubsection{nsteps: $N_{steps}\ge 0$}

Total number of time steps. 


\subsubsection{ntout: $N_{tout}> 0$}

How often to output data.  Since the first time step is always output, 
the total number of time steps that are output is then given
by
\[
1+\mbox{floor}\left(\frac{N_{steps}}{N_{tout}}\right)\,.
\]


\subsubsection{ntprog: $0\le N_{tprog}\le 100$}

Report progress at $N_{tprog}$\% intervals if $N_{tprog}>0$, otherwise do not report
progress.  Either way, progress is reported only if the verbose flag is at least \verb+-v+.

\subsubsection{ntconserve: $0\le N_{tconserve}\le N_{steps}$}

How often to output conservative data, such as mass, volume, and potential energy, into
the file specified by ConserveFile.  Output should be edited to suit user's needs in
the \verb+OutputData+ function in \verb+phys.c+.

\subsubsection{nonhydrostatic: Boolean}

(1): Compute the nonhydrostatic pressure, (0): neglect the nonhydrostatic pressure and
calculate the vertical velocity with continuity.

\subsubsection{cgsolver: No longer used}

\subsubsection{maxiters: $I_{max,H}>0$}

Maximum number of iterations allowed of the free-surface conjugate-gradient solver before
reporting nonconvergence.  If the free-surface solver is not converging in less than 200
iterations then the problem is poorly conditioned.

\subsubsection{qmaxiters: $I_{max,Q}>0$}

Maximum number of iterations allowed of the nonhydrostatic pressure conjugate-gradient solver before
reporting nonconvergence. 

\subsubsection{qprecond: 0,1, or 2}

Type of preconditioner to use for the nonhydrostatic pressure solver.  For a typical environmental
flow with an aspect ratio of $\order{10^2}$:
\begin{enumerate}
\item[0] No preconditioner - 1000s of iterations to converge
\item[1] Diagonal preconditioner - 100s of iterations to converge
\item[2] Block-diagonal preconditioner - 10s of iterations to converge.
\end{enumerate}
There is no significant difference among the preconditioners when the aspect ratio is roughly 1.

\subsubsection{epsilon: $\epsilon_H>0$}

Tolerance for free-surface conjugate gradient solver.  Default: $\epsilon_H=10^{-10}$.

\subsubsection{qepsilon: $\epsilon_Q>0$}

Tolerance for nonhydrostatic pressure conjugate gradient solver.  Default: $\epsilon_Q=10^{-5}$.

\subsubsection{resnorm: Boolean}

Whether or not to normalize the residual when computing tolerance criteria for the conjugate
gradient solvers.  At a given iteration, if the residual is given by $r=b-Ax$, then tolerance is evaluated with:
\begin{enumerate}
\item[0]: error=$r^T r$
\item[1]: error=$\frac{r^T r}{b^T b}$
\end{enumerate}

\subsubsection{relax: No longer used}

\subsubsection{amp}

Amplitude parameter for use in \verb+boundaries.c+ in the form \verb+prop->amp+.

\subsubsection{omega}

Frequency parameter for use in \verb+boundaries.c+ in the form \verb+prop->omega+.

\subsubsection{flux}

Flux parameter for use in \verb+boundaries.c+ in the form \verb+prop->flux+.

\subsubsection{timescale}

Timescale parameter for use in \verb+boundaries.c+ in the form \verb+prop->timescale+.

\subsubsection{volcheck: Boolean}

(0): Do not check for volume conservation, (1): check for volume conservation and print out errors
if volume is not conserved to within the tolerance defined by \verb+CONSERVED+ in \verb+suntans.h+.
Prints regardless of verbosity setting at command line.

\subsubsection{masscheck: Boolean}

(0): Do not check for mass conservation, (1): check for mass conservation and print out errors
if volume is not conserved to within the tolerance defined by \verb+CONSERVED+ in \verb+suntans.h+.
Prints regardless of verbosity setting at command line.

\subsubsection{nonlinear: 0,1, or 2} \label{sec:nonlinear}

Advection scheme for momentum:
\begin{enumerate}
\item[0]: No advection of momentum, i.e. $\ub\cdot\nabla\ub=0$.
\item[1]: First-order upwind for momentum.
\item[2]: Second-order central for momentum.  Note that diffusion must be present for stability.  While
it is difficult to perform an exact stability analysis for the unstructured-grid equations, the 1-d
stability limitation requires that 
grid Peclet number according to 
\[
Pe_\Delta<\frac{2}{C}\,,
\]
where $Pe_\Delta = \frac{\max(|U|)\max(D_g)}{\min(\nu_H)}$ and $C=\frac{\max(|U|)\Delta t}{\max(D_g)}$,
which translates into the requirement that the horizontal diffusion be restricted to
\[
\nu_H>\frac{\max(|U|)^2\Delta t}{2}\,.
\]
Equivalently, the vertical diffusion is restricted by
\[
\nu>\frac{\max(|w|)^2\Delta t}{2}\,.
\]

\end{enumerate}

\subsubsection{newcells: Boolean}

Since the advection of momentum is not conservative in the upper cells, the velocity in the upper cells
can be adjusted to ensure that
\[
U_{top}^{n+1}\Delta z_{top}^{n+1} = 
U_{top}^{n}\Delta z_{top}^{n}\,.
\]
This should be used with caution and results should be checked to make sure this does not alter them
significantly.  Otherwise the upper layer depth is too small relative ot the free surface motions.\\
(0): Do not adjust upper layer cells, (1): Adjust upper layer cells.

\subsubsection{wetdry: Boolean}

Different time-stepping schemes are used for horizontal scalar advection when wetting and drying occur (vertical
advection always uses the $\theta$ method).  To guarantee consistency
with continuity, the $\theta$ method must be used for scalar advection.  This is formally first-order
accurate in time, so results are more accurate when second-order Adams-Bashforth is used if there
is no wetting and drying.  Therefore, the wetdry Boolean is set according to:\\
(0): No wetting and drying, employ AB2 for horizontal scalar advection\\
(1): Wetting and drying: employ $\theta$-method for horizontal scalar advection.\\
Depending on the stratification at the surface and the grid size it is often desirable to employ
\mbox{wetdry = 1} so as to ensure consistency with continuity and avoid spurious oscillations in
the temperature and salinity fields.

\subsubsection{Coriolis\_f: $f\ge 0$}

Coriolis frequency $f=2\Omega\sin(\phi)$, where $\phi$ is the latitude.

\subsubsection{sponge\_distance: $D_{sponge}\ge 0$}

An example of how to implement a sponge layer is shown in \verb+phys.c+ in the form
\[
\frac{u^{n+1}-u^n}{\Delta t} = -\frac{1}{\tau_{sponge}}\exp\left(-\frac{x}{D_{sponge}}\right)\,,
\]
where the $\tau_{sponge}$ timescale is defined by the value of sponge\_decay.  If $D_{sponge}=0$ then
no sponge layer is employed.  {\bf Note that this sponge layer assumes the boundary is located at $x=0$
and that the sponge layer is roughly $D_{sponge}$ m thick and extends in the positive $x$ direction!}

\subsubsection{sponge\_decay: $\tau_{sponge}>0$}

Decay timescale for sponge layer (in seconds).

\subsubsection{readSalinity: Boolean}

(0): Vertical salinity distribution is specified in \verb+initilization.c+\\
(1): Read vertical salinity distribution from the file specified by the InitSalinityFile parameter.
This file must be in binary form and must contain $N_{kmax}$ double precision values in top-to-bottom
order.

\subsubsection{readTemperature: Boolean}

(0): Vertical temperature distribution is specified in \verb+initilization.c+\\
(1): Read vertical temperature distribution from the file specified by the InitTemperatureFile parameter.
This file must be in binary form and must contain $N_{kmax}$ double precision values in top-to-bottom
order.

\subsection{Input/Output files}

These parameters define the names of the input files.  The syntax of the line in \verb+suntans.dat+ is given by
\begin{verbatim}
filespecifier  fileprefix   # Description of parameter
\end{verbatim}
Global files do not contain the processor suffix, while per-processor files contain the processor
number as the suffix.
For example, in a four-processor run, there will be four i/o files relating
to this specifier, i.e. fileprefix.0, fileprefix.1, fileprefix.2, and fileprefix.3.  The file
called fileprefix (with no suffix) contains information for all processors.  

\subsubsection{pslg: input}

Planar straight line graph file. Does not contain the processor number suffix.
 See Section \ref{sec:grids} for details.

\subsubsection{points: input/output}

Delaunay points file ($N_p$ total points).  Does not contain
the processor number suffix. For details see Section \ref{sec:grids}.
\begin{list}{}
\item Size: $N_p\times 3$
\item Type: ASCII
\item Format: (float)xp, (float)yp, (int)marker
\end{list}

\subsubsection{edges: input/output}

Edge connectivity file ($N_e$ total edges). For details see Section \ref{sec:grids}.
\begin{list}{}
\item Size: $N_e\times 5$
\item Type: ASCII
\item Format: 
\begin{list}{}
\item (int$\times$ 2)Indices to Delaunay points that make up endpoints
\item (int)Edge marker for boundary condition
\item (int$\times$ 2)Indices to Voronoi points on either side of edge
\end{list}
\end{list}


\subsubsection{cells: input/output}

Cell connectivity file ($N_c$ total cells). For details see Section \ref{sec:grids}.
\begin{list}{}
\item Size: $N_c\times 8$
\item Type: ASCII
\item Format: 
\begin{list}{}
\item (float$\times$ 2) Voronoi coordinates
\item (int $\times$ 3) Indices to points that make up vertics
\item (int $\times$ 3) Indices to neighboring cells 
\end{list}
\end{list}

\subsubsection{depth: input}

Contains bathymetry data used to interpolate the depth onto the Voronoi points only
if the IntDepth parameter is set to 1.  Otherwise the depth is set in \verb+initialization.c+.
The spacing of the data is arbitrary since the interpolation routine searches for the
nearest neighbors to perform the interpolations.  Note that the absolute value of the depth
is read in from this file, so SUNTANS does not distinguish between elevation and depression.
\begin{list}{}
\item Size: $\mbox{Number of bathymetry data points}\times 3$
\item Type: ASCII
\item Format: 
\begin{itemize}{}
\item (float)x
\item (float)y
\item (float)depth
\end{itemize}
\end{list}

\subsubsection{celldata: input/output}

Contains cell-centered data for each processor. ($N_c$ total cells)
\begin{list}{}
\item Size: $N_c\times 17$
\item Type: ASCII
\item Format:
\begin{list}{}
\item (float)Voronoi point x
\item (float)Voronoi point y
\item (float)Cell Area
\item (float)Voronoi depth
\item (int)Number of vertical levels
\item (3$\times$int)Index to edge faces
\item (3$\times$int)Index to cell neighbors
\item (3$\times$int)Dot-product with outward normal on faces
\item (3$\times$float)Distance from Voronoi point to normal on faces
\end{list}
\end{list}

\subsubsection{edgedata: input/output}

Contains edge-centered data for each processor. ($N_e$ total edges)
\begin{list}{}
\item Size: $N_e\times 13$
\item Type: ASCII
\item Format:
\begin{list}{}
\item (float)Edge length
\item (float)Voronoi length 
\item (float$\times$ 2)x-y components of unique normal
\item (float$\times$ 2)x-y coordinates of center of edge
\item (int)Number of edges in vertical
\item (int)Maximum number of neighboring cells in vertical
\item (int)Upwind cell neighbor (if $U>0$)
\item (int)Downwind cell neighbor (if $U>0$)
\item (int$\times$ 2) gradf pointers (see Section \ref{sec:grids})
\item (int)Edge marker
\end{list}
\end{list}

\subsubsection{vertspace: input/output}

Contains vertical grid spacing in top-to-bottom ordering.  Same for all processors.
\begin{list}{}
\item Size: $N_{kmax}\times 1$
\item Type: ASCII
\item Format: (float)Grid spacing
\end{list}

\subsubsection{topology: input/output}

Contains topology information per processor. ($N_c$ = number of cells, $N_e$ = number of edges)
\begin{list}{}
\item Size: Depends on processor topology
\item Type: ASCII
\item Format:
\begin{list}{}
\item (int)Number of processors
\item (int)Number of neighboring processors (Numneighs)
\item (int$\times$Numneighs)Processor neighbor list
\item For each neighboring processor:
\begin{list}{}
\item (int)Number of cells to send (cellsend)
\item (int)Number of cells to receive (cellrecv)
\item (int)Number of edges to send (edgesend)
\item (int)Number of edges to receive (edgerecv)
\item (int$\times$ cellsend)Indices of cells to send
\item (int$\times$ cellrecv)Indices of cells to recv
\item (int$\times$ edgesend)Indices of edges to send
\item (int$\times$ edgerecv)Indices of edges to recv
\end{list}
\item (int$\times$3)Indices for celldist array
\item (int$\times$6)Indices for edgedist array
\item (int$\times N_c$)Indices for cellp array
\item (int$\times N_e$)Indices for edgep array
\end{list}
\end{list}

\subsubsection{FreeSurfaceFile: output}

Contains the free-surface data. ($N_c$ = number of cells)
\begin{list}{}
\item Size: $\left[1+\mbox{floor}\left(\frac{N_{steps}}{N_{tout}}\right)\right]\times N_c$
\item Type: binary
\item Format: (double$\times N_c$) for each time step.  
\item Matlab equivalent: h(1:Nc) for each time step.
\end{list}

\subsubsection{HorizontalVelocityFile: output}

Contains the three components of velocity at the Voronoi points. ($N_c$ = number of cells)
\begin{list}{}
\item Size: $\left[1+\mbox{floor}\left(\frac{N_{steps}}{N_{tout}}\right)\right]\times N_{kmax} \times 3 \times N_c$
\item Type: binary
\item Format: (double$\times N_{kmax} \times 3 \times N_c$) for each time step.  
\item Matlab equivalent: u(1:Nkmax,1:3,1:Nc) for each time step.
\end{list}

\bigskip
\noindent
Note: empty cells (beneath z=-d) contain value of \verb+EMPTY+ defined in \verb+suntans.h+.


\subsubsection{VerticalVelocityFile: output}

Contains the vertical velocity at the horizontal faces. ($N_c$ = number of cells)
\begin{list}{}
\item Size: $\left[1+\mbox{floor}\left(\frac{N_{steps}}{N_{tout}}\right)\right]\times (1+N_{kmax}) \times N_c$
\item Type: binary
\item Format: (double$\times (1+N_{kmax})\times N_c$) for each time step.  
\item Matlab equivalent: w(1:Nkmax+1,1:Nc) for each time step.  
\end{list}

\bigskip
\noindent
Note: empty cells (beneath z=-d) contain value of \verb+EMPTY+ defined in \verb+suntans.h+.

\subsubsection{SalinityFile,BGSAlinityFile,TemperatureFile,\\PressureFile,VerticalGridFile,EddyViscosityFile,\\ScalarDiffusivityFile: outputs}

Contain data at Voronoi points for the given quantity. ($N_c$ = number of cells)
\begin{list}{}
\item Size: $\left[1+\mbox{floor}\left(\frac{N_{steps}}{N_{tout}}\right)\right]\times N_{kmax} \times N_c$
\item Type: binary
\item Format: (double$\times N_{kmax}\times N_c$) for each time step.  
\item Matlab equivalent: s(1:Nkmax,1:Nc) for each time step.  
\end{list}

\bigskip
\noindent
Note: empty cells (beneath z=-d) contain value of \verb+EMPTY+ defined in \verb+suntans.h+.

\subsubsection{ConserveFile: output}

File used to report conservative variables such as volume, mass, and energy.  The format for this file
is user-implemented in the \verb+OutputData+ function in \verb+phys.c+.

\subsubsection{ProgressFile: output}

Short file containing a printout of the run progress in the form
\begin{verbatim}
On (int) of (int),t=(float) ((int)% Complete, (int) output)
\end{verbatim}
to show total number of time steps complete, the current simulation time (t=...), 
the percentage completion, and the number of steps that have been output for
viewing.

\subsubsection{StoreFile: output}

Saves data to be loaded for a restart run or to analyze data in existence at the time of
a blow up.    
\begin{list}{}
\item Size: Depends on grid geometry since cells beneath the depth are not stored (unlike other
individual data files).
\item Type: binary
\item Format: 
\begin{list}{}
\item (int)last time step
\item (double$\times N_c$)Free surface
\item (double$\times N_e \times N_{ke}$)AB2 term at n-1 for horizontal momentum 
\item (double$\times N_c \times N_{k}$)AB2 term at n-1 for vertical momentum 
\item (double$\times N_c \times N_{k}$)AB2 term at n-1 for salinity
\item (double$\times N_c \times N_{k}$)AB2 term at n-1 for temperature
\item (double$\times N_c \times N_{k}$)AB2 term at n-1 for Mellory-Yamada 2.5 q when turbulence=1
\item (double$\times N_c \times N_{k}$)AB2 term at n-1 for Mellory-Yamada 2.5 L  when turbulence=1
\item (double$\times N_c \times N_{k}$)Mellor-Yamada 2.5 q  when turbulence=1
\item (double$\times N_c \times N_{k}$)Mellor-Yamada 2.5 L  when turbulence=1
\item (double$\times N_c \times N_{k}$)Turbulent eddy-viscosity
\item (double$\times N_c \times N_{k}$)Turbulent scalar-diffusivity
\item (double$\times N_e \times N_{ke}$)Horizontal velocity
\item (double$\times N_c \times (N_{k}+1)$)Vertical velocity
\item (double$\times N_c \times N_{k}$)Nonhydrostatic pressure 
\item (double$\times N_c \times N_{k}$)Salinity
\item (double$\times N_c \times N_{k}$)Temperature
\item (double$\times N_c \times N_{k}$)Background salinity 
\end{list}
\end{list}

\subsubsection{StartFile: input}

Used to read in data for a restart run.  See Section \ref{sec:restart} for details.


\subsubsection{InitSalinityFile: input}

Contains initial vertical top-to-bottom salinity distribution.  Read only when readSalinity=1.
\begin{list}{}
\item Size: $N_{kmax}$
\item Type: binary
\item Format: (double$\times N_{kmax}$)
\end{list}


\subsubsection{InitTemperatureFile: input}

Contains initial vertical top-to-bottom temperature distribution.  Read only when \\readTemperature=1.
\begin{list}{}
\item Size: $N_{kmax}$
\item Type: binary
\item Format: (double$\times N_{kmax}$)
\end{list}
