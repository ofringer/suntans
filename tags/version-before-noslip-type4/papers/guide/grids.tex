\section{Creating or Reading in Triangular Grids} \label{sec:grids}

\subsection{Using the triangle libraries} \label{sec:tri}

Triangular grids can be created from a simple planar straight line
graph (pslg) which is specified as the \verb+pslg+ file in \verb+suntans.dat+.
The format of this file is similar to the format of the planar straight
line graph file for use in triangle.  The planar straight line graph
file is a listing of points and edges that make up a closed contour.
These edges will comprise the boundaries of the triangulation that are
created by the triangle libraries.  The simplest PSLG file is one which
specifies a box with sides of length 1.  This requires four points and
four edges, as shown in the following file \verb+box.dat+:
\begin{verbatim}
# Number of points
4
# List of points (x,y,marker)
0.0 0.0 0
0.0 1.0 1
1.0 1.0 2
1.0 0.0 3
# Number of segments
4
# List of segments and boundary markers (point #, point #, marker)
0 1 1
1 2 1
2 3 1
3 0 1
# Number of holes
0
# Minimum area
.005
\end{verbatim}
The minimum area causes the triangle program to continue to add triangles
until the minimum area of one of the triangles satisfies this area.  
Setting the \verb+pslg+ variable to \verb+box.dat+ in \verb+suntans.dat+
and running SUNTANS with
\begin{verbatim}
mpirun sun -t -g --datadir=./data
\end{verbatim}
This creates a triangulation composed of 256 right triangles within a square
with sides of length 1.  The three columns after the number of points
specification correspond to the x-y coordinates of each point, along
with a marker for each point.  Following the number of segments specification,
the first two columns specify the indices of the points that make up
the end points of each segment in C-style numbering, such that the indices
go from 0 to one less than the number of points.  The third column in the
segment list specifies the marker for each segment.  This is how the boundary
conditions are specified, and is discussed in more detail in Section \ref{sec:boundary}.
The number of holes is currently not used.

Degenerate triangulations may arise when neighboring triangles are close to having
right angles.  In this case the distance between the Voronoi points may be close
to zero and hence may severely limit the time step.  This is remedied by setting
the \verb+CorrectVoronoi+ variable to 1 in \verb+suntans.dat+.  If this
parameter is set, then the Voronoi points are corrected when the distance
between Voronoi points is less than \verb+VoronoiRatio+ times the distance between
the cell centroids.  For example, for two neighboring right triangles, the Voronoi
points will be coincident and hence the distance between the Voronoi points of these
neighboring triangles will be zero.  If \verb+VoronoiRatio+ is set to \verb+0.5+, 
then the Voronoi points are adjusted so that the distance between them is half the
distance between the cell centroids.  If \verb+VoronoiRatio+ is set to \verb+1+,
then the Voronoi points correspond to the cell centroids.

A useful m-file \verb+checkgrid.m+ is provided in the suntans/mfiles directory that
determines the shortest distance between Voronoi points on a grid as well as a histogram
of the Voronoi distances on the entire grid.  It is important to remember that most
grid generation packages, including triangle, are meant for finite-element calculations
in which calculations are usually node-based (i.e. everything is stored at the Delaunay points).
Because SUNTANS is Voronoi-based, this places a strong constraint on the distance between
the Voronoi points and as a result it can be quite difficult to generate good grids in
highly complex domains.  We have found a great deal of success with the grid generation
package GAMBIT from Fluent, Inc. and use it for most of our production-scale SUNTANS runs.

\subsection{Reading triangular grids from a file} \label{sec:readgrid}

Use of the \verb+-t+ flag creates three files specified in \verb+suntans.dat+: 
\verb+points+, \verb+cells+, and \verb+edges+.  By default, these are specified
to be
\begin{verbatim}
points    points.dat
edges     edges.dat
cells     cells.dat
\end{verbatim}
The \verb+points+ file contains a listing of the x-y coordinates of the Delaunay points
in the full triangulation before being subdivided among different processors. 
This file contains three columns although the last column is never used.  The
total number of lines in this file is $N_p$, the number of triangle vertics in the triangulation.
The \verb+edges+ file contains $N_e$ rows each of which defines an
edge in the triangulation, and five columns in the following format:
\begin{verbatim}
Point1 Point2 Marker Voronoi1 Voronoi2
\end{verbatim}
\verb+Point1+ and \verb+Point2+ contain indices to points in the \verb+points+ file
and make up the end points of the Delaunay edges.  Because SUNTANS uses C-style
indexing, then $0\le$\verb+Point1,Point2+$<N_p$.  \verb+Marker+ specifies the type
of edge.  If \verb+Marker=0+, then it is a computational edge, otherwise, it is
a boundary edge, and the boundary condition is specified in Section \ref{sec:boundary}.
The last two entries, \verb+Voronoi1+ and \verb+Voronoi2+, are the indices to the Voronoi
points which make up the end points of the Voronoi edge which intersects this Delaunay edge.
As such, we must have $0\le$\verb+Voronoi1,Voronoi2+$<N_c$.
These Voronoi points correspond to triangles defined in the file \verb+cells+.  Voronoi points
which are ghost points are indicated by a $-1$.
The \verb+cells+ file contains $N_c$ rows each of which corresponds to a
triangle in the triangulation, and 8 columns in the following format:
\begin{verbatim}
xv yv Point1 Point2 Point3 Neigh1 Neigh2 Neigh3
\end{verbatim}
The \verb+xv+ and \verb+yv+ points correspond to the x-y coordinates
of the Voronoi points of each triangle and \verb+Point1+, \verb+Point2+, and \verb+Point3+
correspond to indices to points in the \verb+points+ file which make up the
vertices of the triangle.  These indices must satisfy \\
$0\le$\verb+Point1,Point2,Point3+$<N_p$. \\
\verb+Neigh1+, \verb+Neigh2+, and
\verb+Neigh3+ correspond to indices to neighboring triangles.  Neighboring
triangles which correspond to ghost points are represented by a $-1$.  For neighbors
not lying outside boundaries, we must have \\
$0\le$\verb+Neigh1,Neigh2,Neigh3+$<N_c$.\\
Because SUNTANS determines the number of triangle vertices $N_p$, edges $N_e$, and cells $N_c$
by the number of rows in the \verb+points+, \verb+cells+, and \verb+edges+ files, respectively,
it is important not to have extra carriage returns at the end of these files.

These three files are generated each time the \verb+-t+ flag is used with SUNTANS.  If
the \verb+-t+ flag is not used, then when called with $\verb+-g+$, SUNTANS reads these
three files and computes grid geometry and, if desired, partitions it among several processors.
The \verb+-g+ flag outputs the following data files, which are specified in \verb+suntans.dat+.
One file associated with each of these descriptors is created for each processor in a partitioned
grid.  For example, if the file name specified after \verb+cells+ in \verb+suntans.dat+ is given
by \verb+cells.dat+, then when called with 2 processors, the \verb+-g+ flag would output two files
names \verb+cells.dat.0+ and \verb+cells.dat.1+, each corresponding to the \verb+cells+ file of
each processor.
\begin{itemize}
\item \verb+cells+ Same as the output when using \verb+-t+, except on a per-processor basis.
The indices to the triangle vertices still correspond to indices in the global \verb+points+ file, which is
not distributed among the processors.  All other indices are local to the specific processor.
\item \verb+edges+ Same as the output when using \verb+-t+, except on a per-processor basis.
The indices to the end points of the edges still correspond to indices in the global \verb+points+ file, which is
not distributed among the processors.  All other indices are local to the specific processor.
\item \verb+celldata+ Contains the grid data associated with the Voronoi points of each cell and contains
$N_c$ rows, where $N_c$ is the number of cells on each processor (including interprocessor ghost points).
Each row contains the following entries:
\begin{verbatim}
xv yv Ac dv Nk Edge{1-3} Neigh{1-3} N{1-3} def{1-3}
\end{verbatim}
\begin{itemize}
\item \verb+xv yv+ are the Voronoi coordinates
\item \verb+Ac+ is the cell area
\item \verb+dv+ is the depth at the point \verb+xv,yv+.  This is the depth of the bottom-most
face of the column beneath this cell and is not the actual depth, which is always greater than
\verb+dv+.
\item \verb+Nk+ is the number of vertical levels in the water column.
\item \verb+Edge{1-3}+ are indices to the three edges that correspond to the faces of the cell.
\item \verb+Neigh{1-3}+ are the indices to the three neighboring cells.  
\item \verb+N{1-3}+ is the dot product of the unique normal with the outward normal on each
face.
\item \verb+def{1-3}+ is the distance from the Voronoi point to the three faces.
\end{itemize}
\item \verb+edgedata+ Contains the grid data associated with the Delaunay edges and contains
$N_e$ rows, where $N_e$ is the number of edges on each processor (including interprocessor edges).
Each row contains the following entries:
\begin{verbatim}
df dg n1 n2 xe ye Nke Nkc grad{1,2} gradf{1,2} mark xi{1,2,3,4} eneigh{1,2,3,4}
\end{verbatim}
\begin{itemize}
\item \verb+df+ is the length of the edge.
\item \verb+dg+ is the distance between the Voronoi points on either side of the edge.  If 
this is a boundary edge then \verb+dg+ is twice the distance between the edge and the Voronoi
point on the inside of the boundary.
\item \verb+n1,n2+ are the components of the normal direction of the edge.  These correspond
to the {\it unique} normals of each edge.  The outward normal for this edge corresponding to
a particular cell is given by \verb+n1*N, n2*N+, where \verb+N+ is the dot product of the
unique normal with the outward normal and is specified in the \verb+celldata+ file.
\item \verb+xe, ye+ are the coordinates of the intersection of the edge with the Delaunay
edge.  
\item \verb+Nke+ is the number of active edges in the vertical (see \verb+Nkc+).
\item \verb+Nkc+ is the maximum number of active cells in the vertical which neighbor a given edge.  \verb+Nkc+ is
always at least \verb+Nke+.  See Figure \ref{fig:nkenkc} for a graphical depiction.
\insertfig{.5}{figures/nkenkc}{Depiction of an edge in which Nke=4 and Nkc=6.}{fig:nkenkc}
item \verb+grad{1,2}+ are indices to the Voronoi points defined in the \verb+celldata+ file.
If $N_c$ is the number of cells on a processor, then $0\le$\verb+grad{1,2}+$<N_c$.
\item \verb+gradf{1,2}+ are indices that determine the location of the edge in the ordering
of the \verb+Edge{1-3}+ or \verb+def{1-3}+ arrays.  Each cell contains a pointer to this edge
in its list of \verb+Edge{1-3}+ pointers.  The \verb+gradf{1,2}+ index is a number from 
0 to 2 which determines which face number this edge is of a particular cell.
\item \verb+mark+ Contains the marker type for this edge.  All edges with the value 0 are
computational edges, while other values are described in Section \ref{sec:boundary}.
\end{itemize}
\item \verb+topology+
\begin{verbatim}
Np Nneighs neighbor{0,1,2,...,Np-1}\n
neigh0: num_cells_send num_cells_recv num_edges_send num_edges_recv
cell_send_indices ...
cell_receive_indices ...
edge_send_indices ...
edge_recv_indices ...
neigh1: num_cells_send num_cells_recv num_edges_send num_edges_recv
cell_send_indices ...
cell_receive_indices ...
edge_send_indices ...
edge_recv_indices ...
.
.
.
neigh{Numneighs-1}: num_cells_send num_cells_recv num_edges_send num_edges_recv
cell_send_indices ...
cell_receive_indices ...
edge_send_indices ...
edge_recv_indices ...
celldist[0] celldist[1] celldist[2] ... celldist[MAXBCTYPES-1]
edgedist[0] edgedist[1] edgedist[2] ... edgedist[MAXBCTYPES-1]
cellp[0],...,cellp[Nc-1]
edgep[0],...,edgep[Ne-1]
\end{verbatim}
\item \verb+vertspace+ Contains the vertical grid spacings.  This file has \verb+Nkmax+ rows,
where \verb+Nkmax+ is the number of z-levels.
\end{itemize}
